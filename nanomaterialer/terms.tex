%!TEX root = Nanomat.tex
\ctitletwo{Noen begreper}
\addcontentsline{toc}{section}{NOEN BEGREPER}
\noindent Før vi starter er det greit å være helt klar på hva jeg mener med enkelte begreper som går igjen i teksten:

\paragraph{Partikkel} ``Partikkel'' vil bli brukt som kortform for nanopartikler (av og til også andre partikler som ikke er store nok til å kunne klassifiseres som nanopartikler). Det er \emph{ikke} snakk om elementærpartikler, atomer, ioner eller lignende.

\paragraph{Størrelse} Størrelsen til en nanopartikkel refererer som regel til diameteren til partiklene, hvis de er kulerunde eller tilnærmet kulerunde. Ofte er det ikke så nøye om vi mener diameter, radius, sidelengde eller noe annet, siden veldig mange av resultatene er så omtrentlige at det først og fremst er størrelsesordenen vi er interessert i. I alle tilfeller er størrelsen et slags mål på den gjennomsnittlige lengden man får ved å måle på tvers av partikkelen.

\paragraph{Overflate og bulk} Overflateatomene er det ytre laget av atomer på en gjenstand, som er eksponert for en annen fase. De kan enten ta del i en overflate (for eksempel overflaten mellom fast stoff og gass) eller en grenseflate (mellom to faste stoffer eller to ikke-blandbare væsker). Bulkatomene er de resterende atomene. ``Bulk'' og ``bulkmateriale'' blir også brukt i en annen betydning, nemlig som en motsetning til det nanostrukturerte materialet - bulkmaterialet er materialet i sin vanlige form.

\paragraph{Metastabil} En metastabil situasjon (for eksempel en metastabil fase, et metastabilt materiale, en metastabil krystallstruktur, etc.) er når systemet er i metastabil likevekt - den frie energien til systemet er i et lokalt minimum, men ikke i et globalt minimum. Hvis et system er i en metastabil fase, kan en tilstrekkelig stor forstyrrelse av systemet gjøre at systemet går ut av det lokale minimumet og inn i et annet minimum (enten en stabil fase som \emph{er} et globalt minimum i den frie energien, eller en annen metastabil fase). 
\vfill
\paragraph{Franske ord} Som et tiltak for å bekrefte alle stereotypier om kulturell arroganse har de franske forfatterne av \emph{Nanomaterials and Nanochemistry} valgt å la alle grafene være på fransk. Derfor er det nyttig å vite at 
\begin{itemize}
	\item ``taille'' betyr størrelse.
	\item ``coins'' betyr hjørner.
	\item ``aretes'' betyr kanter.
	\item ``apres'' betyr etter.
	\item ``mesures'' betyr trinn/steps.
	\item ``haine'' betyr hat.
\end{itemize}
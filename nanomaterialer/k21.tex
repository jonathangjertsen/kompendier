%!TEX root = Nanomat.tex
\ctitle{Nanostrukturerte materialer i bulk med pulversintring}
\paragraph{Definisjon av sintring} Sintring er å pakke sammen materiale i pulverform ved hjelp av varme og/eller trykk, uten at man smelter materialet fullstendig. Hvis \emph{deler} av materialet begynner å smelte, kalles prosessen \emph{liquid-phase sintring}. I motsatt fall skjer all massetransport i den faste fasen, og prosessen kalles \emph{solid-state sintering}.

Under sintring diffunderer atomene i materialet på tvers av grensene mellom partiklene og gjør materialet til et sammenhengende stykke. Under sintring fjernes eventuell porøsitet i materialet, derfor bruker vi gjerne sintring hvis vi ønsker å gjøre materialet tettere. Sintring kan også gi nye optiske, mekaniske og andre egenskaper.

Sintring kan gjøres på forskjellige måter:
\begin{itemize}
	\item Man kan bare varme opp materialet i en ovn (dette kalles ``naturlig sintring'')
	\item Man kan varme opp materialet samtidig som man trykker det sammen (dette kalles ``hot pressing'', eller, hvis trykket er det samme fra alle retninger, ``hot isostatic pressing'')
	\item Man kan bruke mikrobølgeovner for å varme opp pulveret raskt og jevnt over det hele
	\item Man kan bruke spark-plasma-sintering (SPS), som det kommer mer om senere.
\end{itemize}

Sintring skjer i tre steg: 
\begin{itemize}
	\item I det første steget rører partiklene på seg under oppvarming, og partiklene gror delvis sammen der de er i kontakt (``neck formation''). I begynnelsen er pulveret svært porøst, med mye luft mellom hver partikkel.
	\item I neste trinn gror partiklene mer sammen og blir til korn med korngrenser der det tidligere var grenser mellom partikler.
	\item I det siste trinnet gror kornene, og korngrensene elimineres ved diffusjon.
\end{itemize}
 
\paragraph{Tetthet vs. kornvekst} Under sintring oppstår to forskjellige fenomen: \emph{komprimering} (``densification''), som vi ønsker oss, og \emph{kornvekst}, som vi vil unngå eller kontrollere (dersom vi vil beholde nanostruktur i pulveret vi begynner med). 

\paragraph{Spark-plasma sintering (SPS)} I spark-plasma sintering kjører man mellom 2000 og 20000 \si{\ampere} gjennom materialet i noen millisekunder, samtidig som man trykker det sammen. Dette gjør at det dannes et plasma mellom partiklene. Plasmaet varmer opp området mellom partiklene og lar dem begynne å gro sammen.

SPS har mange fordeler: det tar kort tid (5-20 minutter vs. 5-20 timer med vanlig sintring), bruker lav temperatur (200-500 \si{\celsius} vs. 1000-2000 \si{\celsius} med vanlig sintring), gjør oppvarmingen mer uniform ved at materialet varmes opp fra innsiden (i vanlig sintring varmes det opp fra utsiden), og gir små korn (i vanlig sintring gror kornene raskere).

Så hvorfor bruker man det ikke hele tiden? Fordi det er dyrt å sette opp. :( Men når man først har satt det opp, koster det ikke så mye å bruke.
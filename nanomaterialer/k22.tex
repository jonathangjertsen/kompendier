%!TEX root = Nanomat.tex
\ctitle{Selvorganisering av nanopartikler}
I dette kapittelet skal vi se på hvordan vi kan bruke selvorganisering for å lage nanostrukturerte materialer med makroskopisk størrelse.  Materialene vi lager vil danne 2D- eller 3D-gittere som er analoge med atomgittere i vanlige materialer, men det er nanopartikler, ikke enkeltatomer, som opptar hver plass i gitteret. Slike materialer kan få egenskaper som er en kombinasjon av egenskapene til de individuelle partiklene og bulkmaterialet.

\paragraph{Lodne nanopartikler} Partiklene vi tar utgangspunkt i i dette kapittelet, er dekket med surfaktanter. Man har altså en kjerne av f.eks sølv, og denne kjernen er dekket av alkylkjeder som stikker ut som en ``pels''. Denne ``pelsen'' skaper attraktive interaksjoner mellom partiklene, og det er i bunn og grunn dette som er årsaken til at selvorganiseringen forekommer.

\cstitle{Metoder for å oppnå selvorganisering}
\paragraph{Betingelser for selvorganisering} Siden materialet organiserer seg selv, er vår jobb å skape de riktige betingelsene for slik organisering. Disse betingelsene er makroskopiske -- vi manipulerer ikke hver enkelt partikkel.

Selvorganisering krever at partiklene man begynner med har svært uniform størrelse -- vi kan ikke forvente at partiklene skal danne regulære mønstre hvis ikke partiklene selv er regulære. Partikler bør ikke avvike fra gjennomsnittet med mer enn ca. 13\%. Vi kan riktignok ha partikler med forskjellig størrelse, men da må man for eksempel ha en bimodal fordeling, der partiklene faller innenfor ett av to størrelsesområder som begge er smale.

Hva slags gitter man ender opp med, avhenger av interaksjonene mellom nanopartiklene. Egenskaper som kan påvirke slike interaksjoner er form, overflateegenskaper, ladning, polariserbarhet, magnetisk dipolmoment og masse.

\cstitle{Kreftene som er involvert i selvorganisering}
De viktigste kreftene for selvorganisering er 1) de som har å gjøre med overflatespenning, og 2) van der Waals-interaksjoner.

\paragraph{Kapillærkrefter og overflatespenning}Hvis man drypper en løsning med sfæriske nanokrystaller av uniform størrelse på et flatt substrat, og så lar løsemiddelet fordampe, vil partiklene -- om man er heldig -- spontant danne en struktur. Et eksempel på en struktur som fort kan oppstå, er et heksagonalt gitter (som er det mest mulig tettpakkede 2D-gitteret man kan lage med kuler). Under fordampning trekkes partiklene sammen mot hverandre på grunn av overflatespenning. En slik prosess vil naturligvis føre til at partiklene danner det mest mulig tettpakkede mønsteret som er mulig.

\paragraph{``Wetting forces''} Dette er en annen konsekvens av overflatespenning. Disse kreftene bestemmer om løsningen kommer til å danne en jevn film over substratet (slik at deponeringen av nanokrystaller blir homogen), eller om det vil dannes dråper på overflaten (slik at deponeringen av nanokrystaller blir inhomogen). Man har en ``spredningsparameter'' (spreading parameter) $S$ som avhenger av de forskjellige overflatespenningene:
\begin{equation}
	S = \gamma_{\text{s}} - (\gamma_{\text{sl}} + \gamma_{\text{l}}),
\end{equation}
der
\begin{itemize}
	\item $\gamma_{\text{s}}$ er overflatespenningen mellom substratet og gass/atmosfære
	\item $\gamma_{\text{sl}}$ er overflatespenningen mellom substratet \emph{når det er dekket med løsningen} og gass/atmosfære
	\item $\gamma_{\text{l}}$ er overflatespenningen mellom løsningen og gass/atmosfære
\end{itemize}
Hvis $S>0$ danner løsningen en jevn film. 

Hvis $S<0$ danner løsningen dråper. Da er det så stor overflatespenning mellom væsken og gass/atmosfære at væsken vil minimere overflateenergien ved å danne konvekse dråper.

\paragraph{van der Waals-interaksjoner} van der Waals-krefter kan også forårsake interaksjoner mellom partikler. Dette er altså krefter mellom elektriske dipoler -- enten permanente dipoler, dipoler som har blitt indusert av andre dipoler, eller dipoler som induseres spontant på grunn av tilfeldige fluktuasjoner i elektrontetthet. Det siste tilfellet er det man kaller London-dispersjonskrefter.

\paragraph{Hvilken betydning har substratet?} Interaksjonene mellom substratet og partiklene kan være attraktive eller frastrøtende:
\begin{itemize}
	\item Dersom substratet har frastøtende interaksjoner med partiklene, vil de attraktive interaksjonene mellom partiklene være av størst betydning. Dette fører til at det dannes tettpakkede lag, gjerne i små flak.
 	\item Dersom substratet har attraktive interaksjoner med partiklene, vil ikke partiklene ha mulighet til å bevege seg like mye rundt på substratet som i forrige tilfelle. Dét fører til dannelse av enkeltlag som er spredt over hele området man dekket med løsning, men enkeltlagene vil ikke være særlig tettpakkede.
 \end{itemize}
Et godt eksempel på balansen mellom partikkel-partikkel interaksjoner og substrat-substrat-interaksjoner er nanokrystaller av sølv dekket med alkantioler (\ce{R-SH}), på et substrat av grafitt. Grafitt har attraktive interaksjoner med slike partikler, så man skulle forvente at partiklene være spredt over et stort område. Dette er sant dersom alkylkedene til alkantiolene er korte (10 eller færre karbonatomer).

\emph{Men}: dersom alkylkjeden er lang (12 eller flere karbonatomer) vil alkylkjeder fra tilstøtende nanopartikler ``flettes'' inn i hverandre -- attraktive interaksjoner mellom alkylkjeder i tilstøtende nanopartikler vil da dominere interaksjonene mellom substratet og partiklene. Dette gjør at partiklene danner et tettpakket, heksagonalt mønster. Dersom alkylkjedene er kortere vil de ikke være lange nok til å flettes i hverandre på samme måte.

\paragraph{Hvilken betydning har fordampningsprosessen? Bernard-Marangoni-ustabiliteter} Hvis løsemiddelet fordamper sakte og gradvis, får man en jevn fordeling av nanopartikler, som man skulle forvente. 

Men hvis løsemiddelet fordamper veldig fort, kan det hende at partiklene samler seg i ringer på noen mikrometers størrelse. Disse dannes ved akkurat samme prosess som dannelse av såpebobler, nemlig \emph{Marangoni-effekten}. Forklaring:
\begin{enumerate}
	\item En væske med høy overflatespenning trekker mer på væske og partikler rundt seg, enn væske med lav overflatespenning.
	\item Hvis det er en gradient i overflatespenningen, vil partiklene gå fra der det er lav overflatespenning til der det er høy overflatespenning.
	\item Overflatespenning avhenger av temperatur: jo høyere temperatur, jo lavere overflatespenning.
	\item Under en rask fordamping er temperaturen ujevn - enkelte steder vil til tider ha litt høyere temperatur enn andre.
	\item Væske og partikler vil bevege seg radielt vekk fra det varmeste området. Dermed dannes det ringer.
\end{enumerate}
For å unngå dette kan vi bruke et løsemiddel som fordamper sakte, for eksempel en litt stor alkan.

\vfill
\cstitle{Krystallstruktur for selvorganiserte materialer}
Selvorganiserte materialer kan danne mange forskjellige strukturer. Hva slags struktur som dannes avhenger av størrelse og form på partiklene, samt interaksjonene mellom partiklene, men også hardhet - hvor mye partiklene kan komprimeres og deformeres - har en betydning.

Selvorganiserte materialer som danner tredimensjonale strukturer, viser seg å lage strukturer som er analoge med krystallstrukturene i vanlige materialer: HCP, FC og BCC er eksempler på strukturer som har blitt observert. 

Når man har nanopartikler med diameter $D_{\text{kjerne}}$, dekket med karbonkjeder med lenge $L$, kan man definere en parameter
\begin{equation}
	\chi = 2\frac{L}{D_{\text{kjerne}}},
\end{equation}
som viser seg å ha betydning for hvilken krystallsturktur som dannes. Dersom $\chi$ er mindre enn ca. 0.73, vil partiklene danne tettpakkede strukturer som FCC og HCP. For større $\chi$ vil man se mindre tettpakkede strukturer som BCC, og blir $\chi$ enda større vil man se enda mindre tettpakkede (og mindre symmetriske) strukturer.

\paragraph{Betydningen av form} Hvis nanopartiklene har en viss form, kan dette påvirke hvilken krystallstruktur man ender opp med. For eksempel vil kubiske nanokrystaller føre til at partiklene danner et kubisk gitter. Dette er fordi karbonkjedene som stikker ut av nanopartiklene flettes best inn i hverandre når de er parallelle -- altså når man har to tilstøtende kubeflater.

\paragraph{Binodale størrelsesfordelinger gir krystallstrukturer som ligner på krystallstrukturene til salter} Hvis du har en blanding av nanopartikler med en binodal størrelsesfordeling -- noen har radius $r$, resten har en større radius $R$ -- så vil de pakke seg sammen på en måte som avhenger av forholdet $r/R$ mellom radiene. Dette minner om hvordan forholdet mellom størrelsen på kationet og anionet i et salt bestemmer hvilken krystallstruktur som er mest stabil, men reglene er ikke \emph{helt} de samme. Man må også ta hensyn til ting som nanopartiklenes ladning. 

For eksempel: du har en blanding av \ce{g-Fe2O3}-partikler med størrelse på \SI{13.4}{\nano\meter}, og \ce{Au}-partikler med størrelse på \SI{5.0}{\nano\meter}, slik at $r/R=0.37$. Det viser seg at denne blandingen gir en \ce{NaCl}-struktur. Men hvis du hadde hatt et salt der størrelsesforholdet mellom kation og anion var 0.37, ville du fått en tetrahedral struktur.

% Soft sphere vs. hard sphere model?
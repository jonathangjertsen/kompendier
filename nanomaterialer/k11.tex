%!TEX root = Nanomat.tex
\ctitle{(og 26) Nanoporøse stoffer}
Nå som vi har sett på superkritiske fluider og hydrotermisk syntese\footnote{Kremt.} er vi klare for å se på noen produkter som kan lages med denne syntesen, nemlig stoffer med bittesmå porer.

\paragraph{Terminologi for nanoporøse stoffer} Et nanoporøst stoff består av et uorganisk ``skjelett'', som regel negativt ladet, som inneholder hulrom eller tuneller. Den negative ladningen til skjellettet balanseres av uorganiske kationer eller vannmolekyler rundt om kring. Terminologien rundt nanoporøse materialer er litt pussig:
\begin{itemize}
	\item \emph{Mikroporøse} materialer har porer med en diameter som er mindre enn \SI{2}{\nano\meter}.
	\item \emph{Mesoporøse} materialer har porer med en diameter mellom \SI{2}{\nano\meter} og \SI{50}{\nano\meter}.
	\item \emph{Makroporøse} materialer har porer med en diameter som er større enn \SI{50}{\nano\meter}.
	\item Alle de ovennevnte tilhører det vi klassifiserer som \emph{nanoporøse} materialer. What?
\end{itemize}

\cstitle{Uordnede nanoporøse stoffer}
Dette er nanoporøse stoffer som ikke har en noen periodisk struktur (over store avstander). Disse kan lages industrielt på en av to måter:
\begin{itemize}
	\item Aggregering og delvis sintring av partikler\footnote{Se kapittel 21 om sintring.}. Her vil porene være hulrommene som gjenstår mellom partiklene. Aggregering og delvis sintring danner hovedsakelig mesoporøse og makroporøse materialer. Jo mindre de opprinnelige partiklene er, jo større vil det spesifikke overflatearealet til produktet bli. Silica-gel og porøse keramer er to eksempler på porøse stoffer som kan lages med en slik prosess.
	\item Materialer der man danner porer ved å selektivt løse opp deler av et opprinnelig ikke-porøst materiale. Et eksempel på et materiale laget med en slik prosess er Vycor, et porøst glass. Dette lages ved å varme opp \ce{Na2O-B2O3-SiO2}-glass (med en passende sammensetning av de tre komponentene) til \SI{1400}{\celsius}, og så kjøle ned til en temperatur mellom \SI{450}{\celsius} og \SI{700}{\celsius}. Hvis man gjør dette riktig, vil materialet separere i to faser gjennom såkalt \emph{spinodal dekomponering} i stedet for den vanlige mekanismen for faseseparasjon (nukleering og kornvekst). Ved spinodal dekomponering skjer faseseparasjonen spontant gjennom hele materialet slik at man får et materiale bestående av to sammenvevde nettverk. Det ene nettverket består av 96\% silica, det andre stort sett av natriumborat. Natriumborat er mer syreløselig enn silica, så etter syrebehandling står man igjen med rent silicaglass som har en ormelignende porestruktur. Størrelsen på porene kan justeres med temperaturen og varigheten til prosessen, samt syrestyrken.
\end{itemize}

\cstitle{Ordnede mikroporøse stoffer: zeolitter}
Av stoffer med periodisk struktur har vi mikroporøse \emph{zeolitter}, samt ordnede mesoporøse stoffer. 

\paragraph{Zeolitter} Zeolitter er et kort ord for krystallinske aluminosilikater. Zeolitter består av et porøst, negativt ladet, uorganisk skjelett av silisium, oksygen og aluminium, som balanseres av vannmolekyler og kationer fra alkali- og jordalkalimetaller som henger i porene. I skjelettet inngår hvert oksygenatom i to tetraedere - ett av dem med \ce{Al} i sentrum og det andre med \ce{Si} i sentrum, som man så vidt kan skimte i Figur~\ref{fig:faujasite}. Skjelettet har en periodisk struktur med hulrom eller kanaler som kan være fra \SI{0.3}{\nano\meter} til \SI{0.8}{\nano\meter} -- det finnes utallige typer zeolitter med forskjellig størrelse og sammensetning av hulrom og kanaler.
\begin{figure}[H]
	\bmd\centering
	\includegraphics[width=0.9\linewidth]{faujasite.png}
	\caption{Atommodell av zeolittarten faujasitt. \ce{Si} i grått, \ce{Al} i blått og \ce{O} i rødt.}
	\label{fig:faujasite}
\emd\end{figure}
Før i tida var begrepet ``zeolitt'' begrenset til \emph{naturlige} krystallinske aluminosilikater, men ordet har i dag en utvidet betydning. I dag kan det også referere til stoffer der noe av silisiumet er byttet ut med et annet grunnstoff. Dette grunnstoffet kan være et trivalent grunnstoff som \ce{Al}, \ce{Fe}, \ce{B} og \ce{Ga} eller et tetravalent grunnstoff som \ce{Ti} og \ce{Ge}.

\paragraph{Syntese av krystalline mikroporøse faste stoffer}
Krystalline mesoporøse faste stoffer lages ved hydrotermisk krystallisering av en gel som inneholder både en væskefase og en fast fase. Reaksjonsmediet inneholder
\begin{itemize}
	\item Kjemiske forløpere til rammeverket i strukturen, altså ting som \ce{Si}, \ce{Al} og/eller \ce{P}.
	\item Inorganiske kationer, og/eller organiske specier, som skal balansere den negative ladningen til rammeverket. De spiller også andre roller: de fyller opp mikroporene, de kan påvirke strukturene som dannes, de kan endre de kjemiske egenskapene til gel-en, og de kan stabilisere byggeblokkene i det inorganiske rammeverket.
	\item Kjemiske forløpere til de mineraliserende agentene, altså ting som produserer \ce{OH-} og/eller \ce{F-}. Disse skal balansere mellomtrinnene under sol-gel-prosessen.
	\item Et løsemiddel, typisk vann.
\end{itemize}

Prosessen innebærer at
\begin{itemize}
	\item Reaktantene blandes grundig.
	\item Blandingen holdes ved romtemperatur en stund (ripening).
	\item Blandingen puttes i en autoklave og varmes opp.
\end{itemize}

\paragraph{Syntese av zeolitter} For zeolitter er betingelsene slik:
\begin{itemize}
	\item Basisk medium.
	\item Temperatur på under \SI{200}{\celsius}.
	\item Relativt lavt trykk. Det autogene trykket i autoklaven er på under \SI{20}{\bar}
	\item Tiden det tar kan være alt fra noen timer til flere dager.
\end{itemize} 
Produktet man da får, må vaskes med destillert vann, filtreres og tørkes. Dette produktet er \emph{ikke} porøst, for porene er fylt opp med de inorganiske og/eller organiske kationene som man brukte til syntesen. Man må derfor etterbehandle produktet for å få tilbake porene. Dette kan innebære å bytte ut kationene med andre kationer som man heller vil ha der,\footnote{Ikke spør meg hvordan man gjør det.} eller ved å varme opp til sånn fem hundre grader sånn at eventuelle organiske stoffer brytes ned.\footnote{Tror jeg. Klarte ikke helt å dekryptere setningen ``Porosity is reinstated after cation exchange (inorganic cations) and/or calcination (ammonium cations and organic species)''.}

\paragraph{Bruksområder for zeolitter} Disse finnes det en del av:
\begin{itemize}
	\item Porene gir zeolitter et stort indre overflateareal, som kan gjøre zeolitter nyttige innen \emph{katalyse}; de brukes i olje- og gassindustrien for separasjon av gass.
	\item Det indre overflatearealet gjør også at zeolitter kan adsorbere mye materiale. Derfor brukes de til å adsorbere illeluktende stoffer som \ce{H2S} og formaldehyd.
	\item Den bestemte størrelsen til porene gjør at zeolitter kan brukes som molekylære siler, for eksempel for å filtrere ammonium.
	\item Porene kan fylles med gjødsel, som så slippes ut over lengre tid. Dette gjør det mulig å bruke gjødsel mer effektivt og med mindre avrenning. 
\end{itemize}

\cstitle{Ordnede mesoporøse stoffer}
\paragraph{Ordnede mesoporøse stoffer} Dette er en type silikater og aluminosilukater der porene fortsatt er veldefinerte og har en smal størrelsesfordeling, men nå er de litt større enn det man kan lage med zeolitter. 

Konkrete eksempler på ordnede mesoporøse stoffer er M41S-familien. Medlemmer av denne familien er MCM-41, som har porer ordnet i et (nesten) regulært heksagonalt gitter; MCM-48, som har kubisk symmetri; og MCM-50, som har lamellær struktur. MCM står for Mobile Composition of Matter, av en eller annen grunn.

\paragraph{Syntese av ordnede mesoporøse stoffer} Syntese av ordnede mesoporøse stoffer ligner veldig på prosessen for å syntetisere zeolitter. Denne typen syntese gjøres imidlertid ved litt lavere temperaturer, typ \SI{25}{\celsius} til \SI{150}{\celsius}. I reaksjonsmediet har man gjerne noen surfaktanter,\footnote{Kvartære ammoniumkationer på formen \ce{C_{$n$}H_{$2n+1$}(CH3)3N+}, hvis du lurte.} der den vanligste kalles CTAB\footnote{Heksadekyltrimetylammonium-bromid.} der lengden på alkylkjeden til surfaktanten bestemmer diameteren til kanalen. 

Det er ingen som egentlig vet hvordan disse mesoporøse stoffene faktisk dannes. Her er et forslag til hvordan det kan foregå:
\begin{enumerate}
	\item Surfaktanter danner sylindriske miceller.
	\item Micellene selvorganiserer seg til heksagonale eller kubiske gitre, eller til lamellære (ormelignende) strukturer, avhengig av prosessen.\footnote{Se kapittel 22 om selvorganisering. Det bør gi litt intuisjon om hva som skal til for å danne et heksagonalt gitter.}
	\item En keramisk forløper binder seg til surfaktanthodene, og ``kondenserer'' der som en keramisk fase på utsiden av micellestrukturen.
	\item Surfaktantene fjernes, slik at det som står igjen er den keramiske fasen med mesoporøs struktur.
\end{enumerate}

\paragraph{Bruksområder for ordnede mesoporøse stoffer} Ordnede mesoporøse stoffer er et nyttig substrat som man kan putte funksjonelle monolag på, for å lage funksjonelle mesoporøse materialer (de kan for eksempel fungere som væskefiltre og sånn).

\cstitle{Ordnede makroporøse stoffer}
Boka sier ingenting om ordnede makroporøse stoffer. Det er jo litt synd, for jeg har hørt at du kan lage stoffer med interessante optiske egenskaper (porestørrelsen begynner jo å nærme seg bølgelengden til synlig lys) og sånn.

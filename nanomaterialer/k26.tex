%!TEX root = Nanomat.tex
\ctitle{Nanoporøse stoffer II - nanoporøse medier}




% No mineralizer!

% Tydeligvis MCM-41 som er kult

% C_{16}TMA - størrelsen på miceller bestemmer størrelsen på porene som dannes

% Vanskelig å finne ut av reaksjonsmekanismene - hva som faktisk skjer under reaksjonen. Det ble først foreslått at mesoporøs struktur kommer direkte fra micellestruktur, men nå tror vi på Frasch-modellen.

% Sylindriske miceller i heksagonale arrays: LCT = Liquid Crystal Templating. Figur viser to forskjellige måter å få til det samme på (legge til SiO2 før eller etter assembly)

% Betingelser for MCM-41-fase: se foil.

% 5-celt diagram: Huo-modellen. I trinn B/C er det ingen kondensering - ingen binding mellom de inorganiske speciene. Det skjer først i trinn D.

% Hva er det som gjør at man får enten B eller C? 
% B: lavt nivå av polykondensasjon og høy ladningstetthet. 
% C: motsatt
% Ladningsbalanse mellom surfaktanter og inorganiske specier spiller en rolle i den endelige strukturen.

% Frasch-modellen:. 
% A: initial cationic micelles. Note that there are some surfactants which do not belong to the micelle
% -> add the silica precursors
% B: Silica precursors form around the micelle. Small exchange of ions, but no formation of structure
% -> reduce the pH and add heat
% C: silica precursors precipitate to become silicious prepolymers that may bind to the free surfactants. 
% D: These prepolymers grow and form polymers that can bind more surfactants and eventually form complete  layers of silica around smaller micelles.
%
% Tabell: bare noen eksempler fra boka. 
%
% Må vite: struktur til zeolitter. Beskriv tetraederne (forrige forelesning) som danner byggestener for zeolitter. Vit at man alltid trenger kationer eller vann for å balansere nettverket. Beskriv kort syntesen. Mange applikasjoner som kan beskrives, de er overalt.
% Krystalline vegger vs. amorfe vegger, ordnede vs. uordnede stoffer. Standard karakteriseringsmetoder kommer *senere*.
















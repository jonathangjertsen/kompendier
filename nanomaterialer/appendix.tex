\appendix
\centerline{\color{Orange}\Huge{Appendiks}}
\begin{changemargin}{2.5cm}{2.5cm}
\section{Bariumtitanat og ferroelektriske tynnfilmer}
\index{ferroelektrisitet}\index{tynnfilmer}\index{bariumtitanat}\index{polarisering}\index{perovskitt}
Et spørsmål man kan gruble over når man leser om endringen i krystallstrukturen til \ce{BaTiO3}, er: hvordan bestemmer \ce{Ti^4+} seg for om den skal gå opp eller ned, når situasjonen er helt symmetrisk over overgangstemperaturen? Svaret fant jeg først etter å ha sendt to meldinger til Reddit-brukeren ``troixetoiles'', som jobber med ferroelektriske tynnfilmer. Hun sendte meg også en forholdsvis utdypende forklaring av hva som skjer i \ce{BaTiO3}, så jeg legger ved samtalen vår her (minus høflighetsfraser).

\paragraph{According to what I've read, this effect ({\ttfamily https://i.imgur.com/75M9M.png}) happens spontaneously when the material is cooled below \SI{120}{\celsius}. I can understand that this would happen if one also applies an electric field as the material cools, but doesn't it also happen in the absence of an external electric field? Does it happen in a specific direction ($P_{\text{up}}$ or $P_{\text{down}}$) every time? If so, what in the world gives rise to that asymmetry in an apparently perfectly symmetric situation?} \mbox{}\\

\noindent So the effect in the picture is the change in crystal structure and symmetry and the accompanying ferroelectric phase transition that happens below the transition temperature. You are right in that it does happens in the absence of an electric field because ferroelectric materials all have a naturally occurring ``spontaneous'' electric polarization, which means it exists in zero field.

Essentially the simplest way to think about what this is (or how I think about it) is a distortion away from the cubic case, which is centrosymmetric. There are two main distortions to the \ce{BaTiO3} unit cell when it become ferroelectric, one is that it elongates in the direction of the polarization, and the other is that the \ce{Ti} atom displaces. Since each atom is really a cation or anion, the displacement of the \ce{Ti} atom causes charges to be displaces from symmetric positions, which leads to the development of a dipole moment in the unit cell. When you add this up over a large number of unit cells in a crystal, then you have a macroscopic ferroelectric polarization.

The question of whether the polarization is constrained to two directions (e.g. $P_{\text{up}}$ and $P_{\text{down}}$) is a good one. And the answer depends on the material system you are talking about. If you look at the drawing above, you probably realize you can rotate the middle or right drawing and get the same structure with P pointed in the four in-plane directions, so you could also get $P_{\text{left}}$ or $P_{\text{right}}$, $P_{\text{front}}$, or $P_{\text{back}}$. In a bulk crystal each of these directions is equally valid because it's all the same distortion of the cubic cell, just in different directions, so in the absence of an electric field one shouldn't be preferred over the others.

A lot of ferroelectric work is done on thin films, however, and this is a different case. For thin films, you grow a very thin (\SIrange{10}{20}{\nano\meter} for example) layer of \ce{BaTiO3} on top of a substrate that usually has the same basic cubic structure but may have different lattice parameters. In a system like this, the in-plane unit cell size of the \ce{BaTiO3} layer is constrained to that of the substrate. So take a system like \ce{BaTiO3} grown on SrTiO3. SrTiO3 has a smaller lattice parameter than \ce{BaTiO3}, so when you grow the latter material on top of the former, you are forcing the in-plane sides of the \ce{BaTiO3} structure to compress in a little bit to match the size of the \ce{SrTiO3} unit cell. This leads to an expansion in the out-of-plane direction that will match what you see in the image you attached. This then constrains the polarization to be in one of two directions: up or down.

Not onto your last query: why would a nicely symmetric system become not-as-symmetric and why would a ferroelectric polarization develop? My studies have been in ferroelectrics similar to \ce{BaTiO3} so I can best speak to materials with the same type of crystal structure. The structure above is called the perovskite structure. You can apply something called a Goldschmidt tolerance factor to these systems, which is an equation that takes the radii of each of the types of atom in the structure and computes how stable the structure is. Depending of the relative sizes of the non-oxygen atoms, structures can be more of less stable, so for \ce{BaTiO3}, at room temperature the \ce{Ti} atom is ``too big'' for the nice cubic structure to be stable. I haven't thought too much about the phase transition between the two states but my first guess at an explanation from this angle would be that due to thermal expansion, as the temperature rises there is more room to accommodate the \ce{Ti} atom so at some point a more symmetric, cubic structure can be stabilized.

\paragraph{What I'm really wondering about is whether the polarization is consistently in one particular direction (say, up) and not the other (down). Because it seems like you could flip that system vertically and get the same structure above the transition temperature, but then below the transition temperature, the \ce{Ti} would go in the opposite direction from what it did last time - even though the conditions didn't actually change.} \mbox{}\\

\noindent You are right in that both directions are equally as favorable energetically. In realistic systems you sometimes have states where the entire sample is in the same polarization, but more often you have cases where the system is in different polarization domains, so different areas of the system are polarized in certain directions. With thin films, as I mentioned before, you are usually constrained to $P_{\text{up}}$ and $P_{\text{down}}$ so you can get areas with both, although the exactly configuration depends on film thickness. That's because if you have these domains, there are walls between them, so you need to figure out which case is more energetically favorable: a single domain, or multiple domains with walls that cost energy. You also need to take other electrostatic stuff into account, too, so it can get a bit complicated, but the take home message is that a system with different polarization domains is very common.

I don't know what journals you have access to, but this is a pretty cool paper on these domains in films thin ferroelectric films where you can get an idea of what they look like: {\ttfamily http://scitation.aip.org/content/aip/journal/apl/93/18/10.1063/1.3013512}

Often in bulk, say in large crystals, you may get domains where the structure goes tetragonal (basically becomes elongated) in different directions, so you can also get different polarization domains. Again, I can't speak to this as much, but with early studies on bulk ferroelectric materials you can get good examples for this.

Once you apply an electric field, however all the polarization wants to align with it, so you create a monodomain state.

I just realized now you also asked about this structure above and below the transition temperature. Above that temperature, the system is no longer in the distorted state. It is cubic and centrosymmetric, so it looks like the leftmost figure in the original picture. Since there are no atomic displacements, there is no polarization. In the original picture the middle and right images are the two stable states below the transition temperature.

\paragraph{So it's pretty much exactly like how you have magnetic domains in ferromagnetic materials? That makes sense. I had the impression that the entire material would somehow polarize as a single domain.} \mbox{}\\
\index{Kittels lov}
\noindent Yep, pretty much the same basic formalism can be used to describe ferroelectric and ferromagnetic domains (Kittel's law). The sizes and energy considerations are different, however, since ferroelectric domain walls tend to be a lot thinner than magnetic ones and other energetics are different, as well, but the two are pretty analogous.

\end{changemargin}

%!TEX root = Nanomat.tex
\ctitletwo{Xia - Endimensjonale nanostrukturer}
\addcontentsline{toc}{section}{XIA - ENDIMENSJONALE NANOSTRUKTURER}
Metodene vi skal se på i dette kapittelet er generelle metoder som er egnet for alle faste stoffer.

\cstitletwo{Template-directed synthesis}
\paragraph{Templat} I ``template-directed synthesis'' har man en templat (eller mal, om du vil) som er et skjelett eller en ramme for strukturen man kan lage. Strukturen man ender opp med vil dannes inni eller rundt templaten, så man vil få en struktur som er komplementær med formen til templaten. 

\paragraph{Fordeler med templater} Vi liker å bruke templater fordi
\begin{itemize}
	\item Det er en enkel metode
	\item Man kan lage strukturer med en veldig kompleks form i ett enkelt trinn
	\item Man kan lage mye om gangen
	\item Det er kostnadseffektivt
\end{itemize}

\paragraph{Ulemper med templater} Men det er noen problemer med templater:
\begin{itemize}
	\item Det er ofte nødvendig å fjerne templaten, og dette må gjøres selektivt (uten å også fjerne strukturen vi har laget). Dette kan gjøres med kjemisk etsing eller varmebehandling.
	\item Strukturene man lager er gjerne polykrystalline
	\item Man får ganske liten mengde per syntese man gjør % what? men high throughput?
\end{itemize}

\cstitletwo{Ting som kan brukes som templater} 
Noen eksempler på templater er:
\begin{itemize}
	\item Nanostrukturer som har blitt syntetisert med andre metoder
	\item Biologiske makromolekyler, for eksempel DNA-kjeder eller virus som har en nyttig form
	\item Selvorganiserte strukturer
	\item Kanalene i et poræst materiale
	\item Skarpe kanter på overflaten av et solid substrat
\end{itemize}
la oss se på hver av dem:

\paragraph{Detaljer på overflaten av et solid substrat} Hvis vi har et solid substrat med en en eller annen regulær struktur, kan vi bruke geometrien til å ``dekorere'' substratet med nye strukture. Se figur i Xia om dette...

\paragraph{Kanaler i porøse materialer} Vi kan enten fylle kanalene helt opp for å få solide rør, eller vi kan fylle dem delvis opp slik at materialet kun vil være på kantene av kanalene. De endelige produktene blir da hule rør. Det porøse materialet kan være
\begin{itemize}
	\item en polymerfilm som har blitt bestrålt med tunge ioner for å danne skader i filmen, som så utvides med kjemisk etsing. Dette danner uniforme, sylindriske porer gjennom filmen, som er spredt tilfeldig utover. Disse kan fjernes ved å bruke høy temperatur.
	\item aluminiumfolie som gjøres til en anode i et surt miljø. Dette danner tettpakkede (heksagonale) sylindriske porer med uniform størrelse. Denne templaten kan etses bort med \ce{NaOH}.
	\item mesoporøse materialer med kanaler på mellom \SI{1.5}{\nano\meter} og \SI{30}{\nano\meter}
\end{itemize}
Med denne metoden kan vi lage mye forskjellig: metaller, halvledere, keramer og organiske polymerer. Porene kan fylles med gassfase-sputring, injeksjon i væskefase eller elektrokjemisk deponering (se kapitlene om tynnfilmer).

\paragraph{Selvorganiserte molekylære strukturer} Vi kan også bruke strukturene som vi lagde i kapittel 18 om kolloider. Ved riktig konsentrasjon danner surfaktantmolekyler spontant rørformede miceller (eller inverse miceller). Med en passende kjemisk reaksjon kan vi fylle opp disse rørene med et annet materiale, og danne nanorør på den måten.

\paragraph{Eksisterende nanostrukturer} Hvis vi har klart å lage en nanostruktur, kan denne kanskje brukes som templat til syntese av nye, mer komplekse strukturer. For eksempel kan man dekke nanotråder med et nytt materiale for å få ``core-shell nanowires''. Hvis vi så fjerner den opprinnelige nanotråden, står vi igjen med et hult nanorør. Vi kan også kjemisk reagere strukturene -- dette har blitt gjort for å omdanne silisium-nanotråder til silisiumoksid-nanotråder (ved oppvarming og oksidering), og for å omdanne karbon-nanotråder til metallkarbid-nanotråder.

Bruk av eksisterende nanostrukturer er også en veldig nyttig metode for å lage nanotråder av et materiale som ikke gror nanotråder naturlig. Så i stedet for å gro nanotråder av materialet fra bunnen av, begynner man med å lage nanotråder av f.eks \ce{MgO} (som er lett å lage), og dekker så disse nanotrådene med materialet man er interessert i. Med denne metoden har man klart å lage kjempemagnetoresistente nanotråder av \ce{La_{0.67}Ca_{0.33}MnO3}.

\cstitletwo{Endimensjonal vekst}
Her skal vi se på metoder for å gro nanotråder ut av et isotropt medium. Metodene vi brukes kalles VS, VLS og SLS og står for henholdsvis ``vapor-solid'', ``vapor-liquid-solid'' og ``solution-liquid-solid''. De er ganske like, forskjellen ligger altså i hvilken fase reaktantene befinner seg i.

\paragraph{Grad av overmetting bestemmer form} Formen på strukturen vi ender opp med i disse metodene, har å gjøre med hvor overmettet løsningen (eller dampen) vi begynner med er:
\begin{itemize}
	\item Lav overmetting gir fibre og hårlignende strukturer.
	\item Medium overmetting gir vekst av bulk-krystallet.
	\item Høy overmetting gir pulver, som dannes ved homogen nukleering.
\end{itemize}

\paragraph{Vapor-solid (VS)} VS heter det det gjør fordi materialet man skal lage nanotråd av begynner i gassfase, og transporteres til substratet. Substratet holdes ved en relativt lav temperatur, slik at materialet kan kondensere og danne nanotråder der. VS kan skje på to måter:
\begin{itemize}
	\item Direkte: materialet man skal lage en nanotråd av, fordampes, før det transporteres til substratet. og kondenserer der.
	\item Indirekte: det dannes forløpere til materialet man skal lage en nanotråd av før det dannes nanotråder.
\end{itemize}
Man vet ikke alltid om metoden man bruker er direkte eller indirekte, fordi man ikke helt vet mekanismen for VS enda.

\paragraph{Vapor-liquid-solid (VLS)}
VLS-vekst er til nå den mest suksessfulle metoden for å lage store mengder av nanotråder med enkrystallin struktur. I denne metoden begynner man med nanostore dråper av katalysatormetall, som man løser opp gassreaktanter i. Hver av dråpene kommer til å korrespondere til én tråd i det endelige produktet. 

Forutsetningen for at dette skal fungere, er at katalysatormetallet kan danne en væske-legering med materialet man ønsker å lage nanotråder av. Den største utfordringen i VLS er altså å finne en katalysator som passer med nanotråd-materialet. %Ideelt sett skal man kunne danne en eutetisk blanding  (påminnelse: ved den eutetiske sammensetningen av en legering kommer hele legeringen til å smelte samtidig (ved en temperatur som kalles den eutetiske tempearturen), i stedet for at én fase smelter først og så den andre).

Mekanismen er enkel: dråpene med katalysatormetall og reaktanter sitter på overflaten av substratet. Det tilføres stadig mer reaktanter, så dråpene er alltid overmettet med reaktant. Dette gjør at reaktanten presipiterer i grenseflaten mellom dråpen og substratet. Presipiteringen fortsetter etter hvert som man tilfører mer reaktant, og man ender opp med at det gror en nanotråd mellom substratet og dråpen. Se fig. 15 A i Xia (det er sikkert også nyttig å prøve å forstå fasediagrammet i fig. 15 B).

Produkter som har blitt laget med VLS:
\begin{itemize}
	\item Elementære halvledere som \ce{Si} og \ce{G}
	\item III-V-halvledere som \ce{GaAs} og \ce{InP}
	\item II-VI-halvledere som \ce{ZnS} og \ce{CdSe}
	\item Binære oksider som \ce{ZnO} og \ce{SiO2}
\end{itemize}
Det er per i dag \emph{ikke} mulig å bruke VLS til å lage nanotråder av metaller og ternære oksider.

\paragraph{Solution-liquid-solid (SLS)}
SLS er som VLS, men som man kanskje skulle forutsi ut i fra navnet bruker man en løsning med forløpere i stedet for en damp. Av en eller annen grunn skal man da bruke metallorganiske forløpere (som dekomponerer til metall pluss organisk biprodukt), og katalysatoren skal være et metall med lavt smeltepunkt som indium, tinn eller vismut.

\paragraph{Bruk av ``capping reagents''} Hvis vi husker tilbake til kapittel 1, så er likevektsformen til en krystall den der overflateenergien er minimert, og dette skjer når det er et visst forhold mellom de forskjellige fasettene i krystallen (Wullfs teorem). Vi kan også se på det med kinetikk: krystallplanene med høyest overflateenergi er de mest reaktive og dermed de som vokser raskest. De krystallplanene som vokser raskest, vil forsvinne i den endelige stukturen, og gjør at de planene som gror sakte, ender opp med å bli fasettene i produktet.

Men hva om vi er interessert i en struktur der det er mer av de mest reaktive krystallplanene? For å endre likevektsstrukturen til krystallene (og dermed strukturen til produktet vårt) kan vi bruke molekyler som modifiserer overflateenergien til visse fasetter. Surfaktanter kan brukes til dette.

\cstitletwo{Andre metoder}
\paragraph{Selvorganisering av kolloidale nanopartikler} Vi kan bruke de selvorganiserende egenskapene til kolloidale nanopartikler (mer om dette i kapittel 22). Dette gjør vi ved å fylle en eller annen template med de kolloidale partiklene. Templaten kan for eksempel være basert på en struktur som vi har mønstret på forhånd. Vi kan også manipulere organiseringen av partiklene med prober, eller ved å påføre elektriske eller magnetiske felt.

\paragraph{Metoder for å redusere størrelsen på 1D-strukturer} Hvis vi har strukturer med en bredde i størrelsesorden 1 til 10 \si{\micro\meter}, og ønsker å redusere størrelsen til ca. 100\si{\nano\meter} eller mindre, finnes det noen forskjellige metoder:
\begin{itemize}
	\item Strekk tråden i lengderetningen. Da blir den lengre og tynnere. Dette krever selvfølgelig at man har et materiale som er mykt nok til å strekkes på denne måten, men man har fått det til med metaller og glass som har blitt varmet opp til litt under smeltepunktet.
	\item Anisotrop etsing av en enkrystall (hm...)
	\item Bruk litografi
\end{itemize}

\cstitletwo{Egenskaper og bruksområder for 1D-nanostrukturer}
\paragraph{Egenskaper for 1D-nanostrukturer} Noen ting som er godt å vite om 1D-nanostrukturer er:
\begin{itemize}
	\item Jo mindre diameteren på nanotråden er, jo lavere smeltepunkt har den (akkurat som at smeltepunktet synker med minkende størrelse for nanopartikler). Når smeltepunktet blir lavere, er det lettere å gjøre varmebehandling for å få defektfrie nanotråder. Det blir lettere å kutte og sveise sammen nanotråder ved temperaturer som ikke skader materialet. Men det lave smeltepunktet kan også bli et problem, fordi det kan hende at nanotrådene brytes opp til kortere segmenter (de kortere segmentene har lavere fri energi, og når smeltepunktet er lavt skal det ikke så mye til for å overkomme energibarrieren det innebærer å bryte opp nanotråden).
	\item Énkrystalline nanotråder har færre defekter per enhet lengde. Siden defekter gjerne er det som fører til mekaniske feil, bør enkrystalline nanotråder være langt sterkere enn bulkmaterialet.
	\item Enkelte metall-nanotråder kan ende opp med å bli til halvledere dersom diameteren blir liten nok.
\end{itemize}

\paragraph{Bruksområder for 1D-nanostrukturer} Disse strukturene er særlig nyttige for å lage presise sensorer. De har ekstremt høyt forhold mellom overflate og volum, og dette gjør at de elektroniske egenskapene til nanotrådene blir ekstremt sensitive til molekyler som adsorberes på overflaten av nanotrådene. 

Et eksempel på dette er nanotråder av enkrystallint tinndioksid (\ce{SnO2})\footnote{Litt bedre forklaring enn den i Xia finnes her: http://nanowires.berkeley.edu/wp-content/uploads/2013/01/058.pdf}.  Enkrystallint \ce{SnO2} er en n-type halvleder med elektrisk ledningsevne som avhenger sterkt av overflatetilstanden, som igjen er påvirket av molekyler som adsorberes på overflaten. Det har seg slik at molekyler av forurensningsgassen \ce{NO2} fanger opp elektroner dersom de adsorberes på overflaten. En annen egenskap ved \ce{SnO2}-nanotråder som vi tar i bruk er at deres elektriske egenskaper er svært sensitive for UV-lys: UV-lys med bølgelengde på \SI{254}{\nano\meter}, har fotonene en del høyere energi enn båndgapet til \ce{SnO2}, så hvis man bruker slikt lys på nanotrådene øker ledningsevnen med 3-4 størrelsesordener. UV-lys gjør også at \ce{NO2} kan desorbere fra nanotråden (hvis vi ikke bruker UV-lys, vil ikke \ce{NO2} forsvinne når de først har satt seg på nanotråden). Dermed kan vi bruke \ce{SnO2}-nanotråder pluss UV-lys til å måle \ce{NO2}. Det som er fint er at vi kan gjøre det ved romtemperatur, mens vi måtte påført temperaturer på minst \SI{300}{\celsius} hvis vi skulle brukt \ce{SnO2} i form av pulver eller tynnfilmer -- ikke særlig gunstig når vi har å gjøre med en gass som ofte dannes i ekspolosive prosesser.

Halvleder-nanotråder finnes også, og de kan kanskje brukes til noe nyttig en dag. Blant annet er det snakk om å lage enkelt-foton-kilder.



%* Nanotråder av III-V-halvledere ved lavere temperaturer
%* Superkritisk fluid som løsemiddel
%* En viktig komponent er den molekylere komponenten hvis konstituerende grupper kan elimineres for å generere en ikke-molekylær enhet som man kan sette sammen nanotråder av
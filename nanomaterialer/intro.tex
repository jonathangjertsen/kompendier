%!TEX root = Nanomat.tex
\noindent Her er et forsøk på å bruke et språk som er mer aktivt og mer hverdagslig\footnote{Og bitrere.} (og dermed nødvendigvis mindre presist) enn det som står i pensumlitteraturen. Det er for det meste en gjenfortelling av pensumlitteraturen, og pensumlitteraturen er visst ikke helt til å stole på (ref. foreleser). Så ikke tro på alt du leser. Særlig \emph{Nanomaterials and Nanochemistry} burde egentlig selges som en sånn pakkedeal (bok + klype salt), da boka er et produkt av mange forskjellige forfattere, og de forskjellige forfatterne refererer ofte til sitt eget arbeid i tekstene sine.

Nummererte kapitler er fra \emph{Nanomaterials and Nanochemistry} (C. Bréchnigac, P. Houdy og M. Lahmani). I de andre kapitlene er forfatteren nevnt i kapitteltittelen. Jeg har endret litt på kapittelrekkefølgen i forhold til forelesningsplanen. Dette har jeg gjort for å være slem. Samtlige figurer er enten (i) laget av meg, (ii) hentet fra Wikimedia Commons og lisensiert som offentlig eiendom, eller (iii) en modifisert utgave av et bilde fra Wikimedia Commons som er lisensiert som offentlig eiendom. Bortsett fra den om nukleering og vekst, men den endres nok i en seinere utgave.

Eventuelle emneansvarlige i Nanomat som kommer over dette dokumentet, bes om å ikke ta alle de passiv-aggressive kommentarene og fotnotene personlig. Ingenting vondt ment. Forfatterne av \emph{Nanomaterials and Nanochemistry}, derimot...

\begin{flushright}Jonathan Reichelt Gjertsen \\ {\ttfamily jonath.re@gmail.com}\end{flushright}
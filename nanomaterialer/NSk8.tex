%!TEX root = Nanomat.tex
\ctitletwo{NS8 Fullerener og karbonnanorør}
\addcontentsline{toc}{section}{NS8 FULLERENER OG KARBONNANORØR}
PR-avdelingen for karbon er veldig flinke. Når noen sier ``nano'', ser de fleste for seg en molekylmodell av grafen, et karbonnanorør eller en buckyball. La oss se litt på hvordan noen av tidenes mest hypede materialer fungerer.

\paragraph{Krystallstrukturer/allotroper for karbon} Det finnes en del forskjellige former for rent karbon.
\begin{itemize}
	\item Grafen: her er hvert karbonatom bundet til tre andre karbonatomer, med en vinkel på \SI{120}{\degree} mellom hver binding. Hvert atom er altså $sp^2$-hybridisert. Dette gir endimensjonale lag som er ett atom tykke.
	\item Grafitt: dette er mange lag av grafen pakket oppå hverandre, med svake bindinger på tvers av lagene. Siden lagene lett glir forbi hverandre, er grafitt mykt. Grafitt forekommer i naturen.
	\item Diamant: her er hvert karbonatom bundet til fire andre karbonatomer i en tetragonal sturktur. Hvert atom er altså $sp^3$-hybridisert. Denne krystallformen er ekstremt hard. Diamant forekommer i naturen som et mineral og er i utgangspunktet gjennomsiktig, men forurensninger kan farge materialet.
	\item Lonsdaleitt: dette er en tredje naturlig form for karbon som dannes ved meteorittnedlag.
	\item Fulleren: dette er når karbonatomene danner lukkede ``baller'', for eksempel det berømte Buckministerfulleren \ce{C60}. \ce{C60} er det minste stabile fullerenmolekylet, men det finnes også større fullerener. Der \ce{C60} ser ut som en fotball med en sekskant som grenser til hver side av hver femkant, har større fullerener flere sekskanter per femkant. Fullerener ble først oppdaget i 1985.
	\item Nanorør er litt som fullerener, men karbonatomene danner tuber i stedet for lukkede baller. Nanorørenes tykkelse på noen nanometer og lengde på noen mikrometer gir dem deres karakteristiske egenskaper. Nanorør kan enten være ``single-walled'' (SWNT på kort) eller ``multi-walled'' (MWNT). For MWNT's består tubene av flere lag utenpå hverandre. Nanorør ble oppdaget i 1991 i et forsøk på å lage fullerener.
	\item I tillegg er amorft karbon - karbon uten noen bestemt krystallstruktur - en ting som finnes.
\end{itemize}
\vfill
\cstitletwo{Fullerener}
\paragraph{Strukturen til fullerener} I alle stabile fullerener er hver femkant fullstendig omringet av sekskanter. Dette er grunnen til at \ce{C60} er det minste stabile fullerenet. De neste fullerenene som er stabile er \ce{C70}, \ce{C72}, \ce{C76}, \ce{C78} og \ce{C84}.

I \ce{C60} er alle karbonatomer ekvivalente. De inngår i to forskjellige typer bindinger: én dobbeltbinding (som skiller to sekskanter) og to enkeltbindinger (som skiller en femkant og en sekskant). Grunnen til at $\pi$-elektronene ikke er fullstendig delokalisert slik at alle bindingene blir ekvivalente, er krumningen til molekylet - bindingene i de $sp^2$-hybridiserte karbonatomene inngår i en pyramide, ikke i et plan. At bindingene ikke er i samme plan, gjør at \ce{C60} ikke er aromatisk.

Større fullerener er mindre symmetriske enn \ce{C60}, og karbonatomene vil være forskjellige (noen vil være i et hjørne mellom to sekskanter og en femkant, andre vil være i et hjørne mellom tre sekskanter).

\paragraph{Produksjon av fullerener} Fullerener produseres slik (se figur 8.4 i NS):
\begin{enumerate}
	\item Man har to grafitt-elektroder i en heliumatmosfære-
	\item De to elektrodene holdes i kontakt med hverandre. Den ene elektroden er veldig spiss, og kontaktflaten mellom de to elektrodene er veldig liten.
	\item Det sendes en strøm gjennom elektrodene. Siden arealet til kontaktflaten mellom elektrodene er veldig liten, blir også resistiviteten i dette området høy - dermed vil det bli svært høy temperatur i dette punktet på grunn av resistiv oppvarming. Denne høye temperaturen er \SIrange{2500}{3000}{\celsius}, høy nok til at grafitten fordamper og blir til plasma. 
	\item Grafitt-plasma kjøler ned i kontakt med helium-atmosfæren og danner et sotete råmateriale som består av fullerener, nanotuber og amorft karbon.
	\item Det er slik at fullerener av færre enn 100 atomer er løselige i aromatiske løsemidler, så disse kan isoleres med passende ekstraksjonsteknikker. Videre separasjon kan gjøres med kromatografi.
\end{enumerate}

\paragraph{Egenskapene til \ce{C60}} \ce{C60} er fullerenet som har blitt studert mest og best, så resten av diskusjonen av fullerener vil dreie seg rundt buckyballer:
\begin{itemize}
	\item Løselighet: \ce{C60} er uløselig i polare løsemidler og tungt løselig i hydrokarboner. De beste løsemidlene for \ce{C60} er aromatiske løsemidler som benzen, toluen og 1-kloronaftalen (der sistnevnte er desidert best).
	\item Fotofysiske egenskaper: \ce{C60} absorberer lys på en ikke-lineær måte; lys med lav intensitet absorberes lite, men hvis lyset er sterkere absorberes det bedre. Grunnen til dette har å gjøre med elektroniske tilstander og symmetri, og det er nok best å ikke tenke for mye på det. Effekten kan brukes til å beskytte optiske sensorer som kamera og øyne: \ce{C60} kan absorbere laserlys på avveie, men samtidig slippe gjennom belysningen i rommet.
	\item Elektrokjemiske egenskaper: \ce{C60} er en elektronakseptor og kan ta opp opp til 6 elektroner for å danne et stabilt \ce{C60^6-}-anion. \ce{C60} er lett å redusere og vanskelig å oksidere.
	\item Kjemiske egenskaper: mange derivater av \ce{C60} har litt laget ved å ``pode inn'' molekyler på overflaten av molekylet gjennom nukleofile addisjonsreaksjoner. Det vanligste er sykloaddisjon, altså at man lar en av kantene i \ce{C60} bli en del av en ring i reaksjonsproduktet. Slike reaksjoner gjøres ofte for å øke løseligheten til \ce{C60} slik at molekylet blir lettere å manipulere.
\end{itemize}

\cstitletwo{Karbonnanorør}
\paragraph{Strukturen til nanorør} Nanorør har som nevnt en lengde på noen mikrometer og en diameter på \SIrange{1}{10}{\nano\meter}. Krystallstrukturen er som et opprullet grafenlag (det er ikke nødvendigvis slik de oppstår, men det hjelper å tenke på det slik). I hver ende av nanorøret er det femkanter som fører til en krumning som lukker nanorøret. Nanorør kan ha tre forskjellige former, som avhenger av hvordan man ruller sammen grafenlaget (altså hvordan man kobler sammen sekskanter på motsatt ende avdet opprinnelige grafenlaget). De tre formene heter \emph{zigzag}, \emph{armchair} og \emph{chiral}, og forklares best ved å se på Figur 8.9 i NS.

% Kiral vektor, nevn hva m og n er

\paragraph{Elektronisk struktur for grafen og nanorør} Siden nanorør konseptuelt er laget ved å rulle opp grafen, kan de elektroniske egenskapene til nanorør utledes fra de elektroniske egenskapene til grafen. Dette gjøres ved hjelp av kvantemekanikk som er litt i overkant heftig for dette kurset, men det har seg slik at valensbåndet og ledningsbåndet er i kontakt med hverandre ved visse punkter i grafengitteret. Dette gjør grafen til et halvmetall.

Med litt mer av nevnte heftige kvantemekanikk viser det seg at
\begin{enumerate}
	\item Tuber i armchair-konfigurasjon er metalliske.\footnote{Jeg liker å huske dette ved å se for meg en skikkelig mettall type som sitter i en \emph{alt} for stor og myk ørelappstol.}
	\item Tuber i zigzag-konfigurasjon er metalliske hvis $n$ er et heltall ganger 3, og ellers halvledere.
	\item Tuber i kiral konfigurasjon er metalliske hvis $n-m$ er et heltall ganger 3, og ellers halvledere.
\end{enumerate}

\paragraph{Høy-temperatur-syntese av karbonnanorør} Det finnes to kategorier av metoder for å syntetisere karbonnanorør, som skilles fra hverandre ved hvilken temperatur syntesen gjøres ved. Høy-temperatur-syntese av karbonnanorør innebærer å \emph{sublimere grafitt} ved temperaturer som er høyere enn \SI{3200}{\celsius}, og deretter la karbonatomene kondensere i et område med en kraftig temperaturgradient. Høy-temperatur-syntese av karbonnanorør er ganske hardcore:
\begin{itemize}
	\item Du kan lage en kraftig elektrisk lysbue mellom to grafitt-elektroder, slik at det dannes et plasma med en temperatur på nærmere \SI{6000}{\celsius} ved anoden. Plasmaet farer bort til katoden og kondenserer der. Hvis katoden består av grafitt, får man kun MWNT's. For å få SWNT's må man også ha en metallkatalysator i katoden.\footnote{Lurer på hvorfor.}\footnote{Uansett, for å huske på dette liker jeg å se for meg at han mettall-fyren i ørelappstol er singel.}
	\item Du kan bombardere grafitt med høyenergetisk laserstråling. Hvis laserstrålingen ikke er kontinuerlig, men kommer i pulser, vil det dannes små klynger av karbon som kan bli til små SWNT's. Med kontinuerlig stråling kan man lage både MWNT's og SWNT's, da blir mekanismen omtrent den samme som hvis man skulle bruke elektrisk lysbue. For å lage MWNT's sikter man laseren på ren grafitt, for å lage SWNT's sikter man på grafen dopet med en metallkatalysator.
\end{itemize}
\vfill
\paragraph{Lav-temperatur-syntese av karbonnanorør: CVD} Hvis man ikke bruker høy nok temperatur til å sublimere grafitt, kalles metoden lav-temperatur-syntese. Metoden vi skal se på er ``Chemical Vapor Deposition'' (også kjent som CVD på kort, og som ``dette kommer vi tilbake til i kapittel 25'' på praktisk). Her er trinnene for CVD-produksjon av karbonnanorør:
\begin{enumerate}
	\item Du har en gass som inneholder karbon, f.eks \ce{CO} eller metan eller noe sånt, som er en kjemisk forløper til karbonnanorørene. Putt dette i en ovn med temperatur på 500-\SI{1000}{\celsius}.
	\item Du lar molekyler fra denne gassen brytes opp på overflaten av noen nanometer store metall-katalysator-partikler (av \ce{Fe}, \ce{Ni} eller \ce{Co}), slik at karbonet presipiterer på overflaten av partikkelen.
	\item Etter hvert som karbonet presipiterer, vil det danne nanorør! De kan være SWNT's eller MWNT's avhengig av reaksjonsbetingelser som temperatur, trykk og størrelse på nanopartikler. SWNT's får du med \emph{høy temperatur} og \emph{små metallpartikler}.\footnote{Så denne single fyren er \emph{hot}, men han har... små ``metallpartikler''? Nei, nå har denne huskeregelen gått for langt.}
\end{enumerate}
Mekanismen for hvordan tubene vokser, ligner veldig på mekanismen for VLS, som vi skal se på i ``Xia: Endimensjonale nanostrukturer''-kapittelet. Hvis du ikke har sett på dét kapittelet ennå kan du gjøre det nå, det er ganske gøy.

\paragraph{Egenskapene til nanorør} Nanorør har mange bra egenskaper:
\begin{itemize}
	\item \emph{Elektriske egenskaper}: De kan lede strøm ganske bra. Man kan oppnå en strømtetthet på så mye som \SI{10}{\giga\ampere\per\centi\meter\squared}, \footnote{\emph{10 gigaampere}! På halvparten av arealet til en femtiøring! Til sammenligning er strømmen som lager jordas magnetfelt, på 6 GA.} som er minst to størrelsesordener høyere enn man kan oppnå med metaller.
	\item \emph{Mekaniske egenskaper}: Grafitt er veldig stivt hvis du drar i en retning som ligger i planet, ellers er det ikke så veldig stivt. Tilsvarende er nanorør ekstremt stive hvis du drar på dem i lengderetningen\footnote{Knis.}, men de er også veldig lette å bøye, deformere og vri på\footnote{Æææ.}. Dette gjelder først og fremst SWNTs; MWNTs har ikke de samme elastiske egenskapene.
	\item \emph{Kjemiske egenskaper}: Man kan fylle nanorør med molekyler som fullerener eller andre ting for å lage fylte nanorør\footnote{``Fylte nanorør'' er forøvrig det som skal bli signaturretten når jeg avslutter min akademiske karriere for å starte en nanobasert temarestaurant! Den skal hete ``Nanomat''! Kickstarter-kampanje kommer snart.}. Man kan også putte molekyler på overflaten av nanorør.
\end{itemize}

\paragraph{Bruksområder for nanorør} Her er noen bruksområder for nanorør:
\begin{itemize}
	\item Man kan lage bittesmå ohmiske ledninger med dem.
	\item Man kan bruke dem som support for katalytiske metall-nanopartikler!
	\item Man kan bruke dem i solceller!
	\item Man kan putte dem på spissen av en AFM-probe for å få en mye spissere tupp som gir mindre konvolusjon!
\end{itemize}
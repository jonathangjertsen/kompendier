%!TEX root = Nanomat.tex
\ctitletwo{Nanotoksikologi}
\addcontentsline{toc}{section}{NANOTOKSIKOLOGI}
Nanotoksikologi handler om hvordan materialer kan bli giftige når de er i form av ultrafine partikler. Dette området er i dag (2015) preget av at det har blitt gjort veldig lite forskning på nanotoksikologi.

\cstitletwo{Hvorfor kan nanopartikler være skadelige?}
\paragraph{Nanopartikler er små} Det har seg slik at nanopartikler er små. Det er i hovedsak to aspekter ved dette som kan gjøre dem skadelige. Det første er ganske enkelt at de kan komme seg inn på steder der større partikler ikke får tilgang. Det andre er, som vi har sett mange ganger tidligere, at materialer blir mer reaktive av å være i nanopartikkelform, fordi de får større overflate per enhet volum. Denne økte reaktiviteten gjør at giftige materialer kan bli enda giftigere i nanopartikkel-form. Andre faktorer som kan spille en rolle er endringer i materialenes egenskaper (løselighet, form, ledningsevne) når partiklene blir veldig små.

\paragraph{Det vi vet så langt: direkte eksponering} Det som har blitt gjort av forskning, handler for det meste om effektene av direkte eksponering til nanopartikler. Nanopartikler penetrerer og angriper faktisk deler av levende organismer som større partikler ikke kommer frem til. Dette har man funnet ut ved å pumpe \ce{TiO2}-partikler direkte inn i luftveiene til rotter og mus. Partikler med diameter på \SI{20}{\nano\meter} førte til en langt større grad av betennelse (målt som konsentrasjon av hvite blodceller) enn partikler med diameter på \SI{250}{\nano\meter}, per gram \ce{TiO2}. Hvis man ser på grad av betennelse målt mot det samlede overflatearealet av \ce{TiO2}, så ser det ikke ut til å ha noe å si hvor store partiklene er.

I studiene som har blitt gjort så langt er ofte dosene mye høyere enn det er realistisk å oppleve i virkeligheten -- de er store nok til å fullstendig overvelde organismenes forsvarsmekanismer. Sannsynligvis vil organismer reagere kvalitativt annerledes på lave doser av nanopartikler.

\paragraph{Det vi ikke vet: effekt på miljøet} Hvilken effekt har nanopartikler på miljøet --- i jorden, planter, dyr og mennesker? Hvilke risikoer oppstår når nanopartikler fra forbrukervarer lekker ut i miljøet og senere når oss på mer indirekte vis? Hvordan og hvor i økosystemene akkumulerer nanopartikler? Hvordan kan de degraderes?

\vfill
\columnbreak

\cstitletwo{Eksponeringsveier}
Hvordan kan nanopartikler komme seg inn i oss?
\begin{itemize}
	\item \emph{Inhalering} av luftbårne partikler er den viktigste eksponeringsveien, og også den som det har blitt forsket mest på. Partikler med størrelse på ca. \SI{1}{\micro\meter} er de som går lettest gjennom nesa/svelget (større partikler blir blokkert, mindre partikler blir absorbert) og inn i luftrøret og videre inn i lungene.
	\item \emph{Opptak gjennom huden}: nanopartikler kan komme seg gjennom huden enten ved å gå gjennom cellene, ved å gå mellom cellene eller å gå inn langs hårsekkene. Det har blitt vist at nanopartikler kan komme seg inn i blodet og nervesystemet hvis de først kommer gjennom huden. Mange kosmetiske produkter inneholder nanopartikler, men disse er trolig for store til å penetrere huden. Men hva om de brukes på sprukket hud og sår?
	\item \emph{Inntak gjennom mat og drikke}: det har bare blitt gjort noen få studier på inntak av nanomaterialer gjennom fordøyelsessystemet, men det ser ut til at de går rett gjennom og elimineres raskt.
	\item \emph{Injeksjon}: hvis nanopartikler skal brukes til drug delivery, vil naturligvis injeksjon direkte inn i blodet være en eksponeringsvei for nanopartikler.
\end{itemize}

\cstitletwo{Risikovurdering av nanopartikler}
Per 2015 har man dessverre ikke klart å gjøre noe særlig systematisk utredning av risiko knyttet til nanomaterialer og -partikler. Dermed har man ikke grunnlag for å gjøre en tilstrekkelig risikovurdering. 

Om man ikke skal stoppe forskning på nanoteknologi og nanopartikler helt for å være føre var, bør man i hvert fall gjøre en tilstrekkelig innsats i å identifisere og minimere faremomenter. De som produserer nanomaterialer må være klar over hvilken risiko som er forbundet med nanomaterialer, og de må kunne vise at produktene de lager, er trygge nok. Hvis dette ikke blir demonstrert i tilstrekkelig grad, vil offentligheten se på nanomaterialer som farlige, noe som kan føre til feilkarakterisering og unødvendig streng regulering av nanoprodukter. Dette vil igjen være en hindring for forskning på, og kommersialisering av, nanomaterialer.
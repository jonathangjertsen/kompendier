\ctitle{Magnetostatikk}
\cstitle{Magnetisk kraft, strømelement, Biot-Savarts lov}
\noindent Den magnetiske kraften mellom to punktladninger som beveger seg i et vakuum er
\begin{equation}
	\vF=Q_2\vv_2\times\left(\frac{\mu_0}{4\pi}\frac{Q_1\vv_1\times\hat{\vec{R}}}{R^2}\right)
\end{equation}
der $R$ er avstanden mellom dem og $\hat{\vec{R}}$ peker fra ladning 1 til ladning 2. Tilsvarende det elektriske feltet definerer man en magnetisk flukstetthet $\vB$ på grunn av en ladning $Q_1$ med hastighet $\vv_1$ som 
\begin{equation}
	\vB=\frac{\mu_0}{4\pi}\frac{Q_1\vv_1\times\hat{\vec{R}}}{R^2}
\end{equation}
slik at
\begin{equation}
	\vF=Q_2\vv_2\times\vB.
\end{equation}
Magnetfelt kan superponeres,
\begin{equation}
	\vB=\frac{\mu_0}{4\pi}\frac{\sum_{i=1}^m Q_i\vv_i\times\unit{R}_i}{R_i^2}.
\end{equation}
Vi kjenner som regel ikke hastigheten til de ulike ladningene, men strømmen som den samlede bevegelsen av ladninger genererer. Denne strømtettheten er generelt
\begin{equation}
	\vec{J}\dv=\sum_{i=1}^n(N_i\dv)Q_i\vv_i,
\end{equation}
der $n$ er de forskjellige typene ladningsbærere. Denne summen kan i stedet uttrykkes som en sum over alle $m$ ladninger,
\begin{equation}
	\vec{J}\dv=\sum_{i=1}^mQ_i\vv_i,
\end{equation}
slik at
\begin{equation}
	\vB=\frac{\mu_0}{4\pi}\frac{\vec{J}\dv\times\unit{R}}{R^2}
\end{equation}
Ved å integrere opp $\vB$-feltet fra en kontinuerlig romlig fordelt strømtetthet bestående av slike \emph{strømelement}er $\vec{J}\dv$ får vi Biot-Savarts lov,
\begin{equation}
	\vB=\frac{\mu_0}{4\pi}\iiint_v\frac{\vec{J}\dv\times\unit{R}}{R^2}.
\end{equation}
Hvis strømmen er fordelt på en tynn linje er $\vec{J}\dv=\vec{J}\dSs\dls=J\dSs\dl=I\dl$, så
\begin{equation}
	\vB=\frac{\mu_0}{4\pi}\oint_C\frac{I\dl\times\unit{R}}{R^2}
\end{equation}
Hvis strømmen er fordelt på en flate definerer vi en flatstrømtetthet slik at $I\dl=\vec{J}_s\dx\dl=\vec{J}_s\dSs$ og
\begin{equation}
	\vB=\frac{\mu_0}{4\pi}\iint_S\frac{\vec{J}_s\dSs\times\unit{R}}{R^2}.
\end{equation}

\cstitle{Magnetiske krefter og moment}
\paragraph{Lorentz-kraft} Kraften som virker på en ladning som beveger seg i et område med både $\vE$- og $\vB$-felt kan oppsummeres i Lorentz-kraften
\begin{equation}
	\vF=Q(\vE+\vv\times\vB)
\end{equation}

\paragraph{Magnetisk moment} Den magnetiske kraften på et strømelement er
\begin{equation}
	\dF=\sum_{i=1}{m}Q_i\vv_i\times\vB=\vJ\dv\times\vB
\end{equation}
eller for en linjestrøm
\begin{equation}
	\dF=I\dl\times\vB,
\end{equation}
slik at netto kraft på en strømsløyfe blir
\begin{equation}
	F=\oint_C I\dl\times\vB.
\end{equation}
I et uniformt $\vB$-felt blir den netto kraften på sløyfa null,
\begin{equation}
	F=\oint_C I\dl\times\vB=\left(\oint_C I\dl\right)\times\vB=0,
\end{equation}
men dette er bare summen av krefter. Det finnes likevel krefter lokalt som vil komprimere, strekke og rotere sløyfa. Se figur 3.5  i JS for en utledning av at momentet på en (rektangulær) sløyfe blir
\begin{equation}
	\vT=I\vS\times\vB.
\end{equation}
Det magnetiske momentet defineres som
\begin{equation}
	\vm=I\vS,
\end{equation}
så
\begin{equation}
	\vT=\vm\times\vB.
\end{equation}
Dette gjelder også for vilkårlige sløyfer i et uniformt $\vB$-felt.

\cstitle{Magnetisk fluks og vektorpotensial}
\p{Magnetisk fluks} Den magnetiske flyksen gjennom en flate er definert som 
\begin{equation}
	\Phi_S=\iint_S\vB\cdot\dS.
\end{equation}
For en lukket $S$ er $\Phi_S$ alltid 0,
\begin{equation}
	\oiint_S\vB\cdot\dS=0,
\end{equation}
som ved divergensteoremet er ekvivalent med at
\begin{equation}
	\nabla\cdot\vB=0
\end{equation}
Se s. 59 i JS for en utledning av dette. Det er ekvivalent med at magnetiske flukslinjer alltid biter seg selv i halen, det vil si at de ikke kan starte noe sted.

\p{Vektorpotensial} Siden $\nabla\cdot\vB=0$ kan $\vB$ representeres som curlen til et vektorpotensial $\vA$,
\begin{equation}
	\vB=\nabla\times\vA
\end{equation}
Hvis vi antar at
\begin{equation}
	\vA=\frac{\mu_0}{4\pi}\frac{\vJ\dv}{r},
\end{equation}
får vi Biot-Savarts lov ut av det,
\begin{align}
	\vB=\nabla\times\vA&=\frac{\mu_0}{4\pi}\nabla\left(\frac{1}{r}\right)\times(\vJ\dv)\\&=\frac{\mu_0}{4\pi}\frac{(-1)}{r^2}\unit{r}\times(\vJ\dv)\\&=\frac{\mu_0}{4\pi}\frac{\vJ\dv\times\unit{r}}{r^2},
\end{align}
så da sier vi at det vi antok var riktig. Da kan vi superponere for å få at
\begin{equation}
	\vA=\frac{\mu_0}{4\pi}\iiint_v\frac{\vJ\dv}{r}
\end{equation}
eller
\begin{equation}
	\vA=\frac{\mu_0}{4\pi}\oint_C\frac{I\dl}{r}.
\end{equation}
Dette gidder vi å styre og herje på med fordi Biot-Savarts lov for $\vA$ er mye enklere enn den tilsvarende loven for $\vB$, og det forenkler en del beregninger.

\cstitle{Ampères lov for konstante strømmer}
\noindent Amperes lov er at sirkluasjonen av flukstetthet rundt en lukket sløyfe er proporsjonal med strømmen som går gjennom en flate begrenset av sløyfa,
\begin{equation}
	\oint_C\vB\cdot\dl=\mu_0I_{\text{total gjennom }S}=\mu_0\iint_S\vJ\cdot\dS,
\end{equation}
som kan bevises med Biot-Savarts lov og et argument for strukturen til $\vA$, se JS s. 61. Stokes' teorem gir ligningen på differensialform,
\begin{equation}
	\nabla\times\vB=\mu_0\vJ.
\end{equation}

\cstitle{Magnetiske felt i materialer}
\noindent Materialer inneholder mikroskopikse strømsløyfer i form av partikler eller domener med kvantemekanisk spinn som orienterer seg etter den magnetiske flukstettheten og produserer et eget $\vB$-felt som respons på et ytre påtrykt $\vB$-felt. Disse strømsløyfene har magnetisk dipolmoment
\begin{equation}
	\vm=I\vS_m,
\end{equation}
og vi definerer en magnetiseringsvektor tilsvarende polariseringsvektoren i elektrostatikken,
\begin{equation}
	\vM\dv=\sum_{\dv}\vm,
\end{equation}
som hvis alle strømsløyfene er identiske gir at $\vM=N\vm$. Dette er ikke tilfellet, men magnetiseringen kan representeres slik. Hvis vi lar $\vJ$ herfra stå for fri strøm (som ikke er bundet i magnetiske dipoler), blir
\begin{align}
	\oint_C\vB\cdot\dl&=\mu_0I_{\text{total gjennom S}}\\&=\mu_0\left(\iint_S\vec{J}\cdot\dS+I_{\text{gjennom S pga. magn. dipoler}}\right).
\end{align}
Ved å se at kun strømsløyfene som går rundt $C$ kan bidra til det andre leddet, og at disse strømsløyfene bidrar til det andre leddet i den grad de har et dipolmoment som er parallellt med $\dl$, får vi at $I_{\text{gjennom S pga. magn. dipoler}}=\oint_C\vM\cdot\dl$, så
\begin{equation}
	\oint_C\vB\cdot\dl=\mu_0\left(\iint_S\vJ\cdot\dS+\oint_C\vM\dl\right),
\end{equation}
altså er
\begin{equation}
	\oint_C\left(\vB/\mu_0-\vM\right)\cdot\dl=\iint_S\vJ\cdot\dS,
\end{equation}
så definerer vi
\begin{equation}
	\vH=\frac{\vB}{\mu_0}-\vM
\end{equation}
slik at
\begin{equation}
	\oint_C\vH\cdot\dl=\iint_S\vJ\cdot\dS
\end{equation}
eller på lokal form via Stokes' teorem,
\begin{equation}
	\nabla\times\vH=\vJ.
\end{equation}
Merk at det her er $\vB$ som blir analog med $\vD$, mens $\vH$ er analog med $\vE$.

For lineære, isotrope, homogene materialer er
\begin{equation}
	\vM=\chi_m\vH
\end{equation}
så
\begin{equation}
	\vB=\mu_0(\vH+\vM)=\mu_0(1+\chi_m)\vH=\mu_r\mu_0\vH=\mu\vH
\end{equation}

\cstitle{Magnetiske materialer}
\noindent Magnetiske materialer tilhører en av disse kategoriene:
\begin{enumerate}
	\item Diamagnetiske materialer, der $\mu_r<1$, men $\mu\approx 1$
	\item Paramagnetiske materialer, der $\mu_r>1$, men $\mu\approx 1$
	\item Ferromagnetiske materialer, som er ikke-lineære, men som man definerer en slags gjennomsnittlig effektiv permeabilitet for. Denne er mye større enn 1.
\end{enumerate}
I ferromagnetiske materialer avhenger $\vM$ av både det påtrykte feltet og historikken til det påtrykte feltet. Hvis det påtrykte feltet varierer periodisk vil magnetiseringen følge en lukket kurve som vi kaller en hysteresekurve, se figur 3.17 i JS.

\cstitle{Grensebetingelser for B og H}
\noindent Tilsvarende beregninger som da vi regnet grensebetingelser for $\vD$ og $\vE$ gir at
\begin{equation}
	B_{1n} = B_{2n}
\end{equation}
og
\begin{equation}
	\vH_{1t}-\vH_{2t}=\vJ_s\times\unit{n},
\end{equation}
som i spesialtilfellet der det ikke er noen flatestrøm blir til
\begin{equation}
	\vH_{1t}=\vH_{2t}
\end{equation}
Se s. 73 i JS.

\cstitle{Magnetiske kretser}
\noindent En viktig konsekvens av grensebetingelsene for $\vB$ og $\vH$ er at magnetiske feltlinjer liker å holde seg inne i materialer med høy $\mu_r$. At dette er tilfelle kan vi se ved å se på størrelsen til $\vB$ på hver side av en grenseflate mellom et lineært medium med $\mu_r\gg 1$ og vakuum, der det ikke går noen strøm. Hvis vakuum er medium 1, gir Pythagoras at
\begin{equation}
	|\vB_1|^2=B_{1n}^2+B_{1t}^2=B_{1n}^2+\mu_0^2H_{1t}^2=B_{2n}^2+\mu_0^2H_{2t}^2
\end{equation}
Tilsvarende utregning for medium 2 blir
\begin{equation}
	|\vB_2|^2=B_{2n}^2+B_{2t}^2=B_{2n}^2+\mu_0^2\mu_r^2H_{2t}^2
\end{equation}
Fordi $\mu_r\gg 1$, blir $|\vB_2| \gg |\vB_1|$ hvis normalkomponentene ikke er for store. Dette gjør at vi kan anse $\mu$ som en slags ``konduktivitet'' for magnetiske kretser, dersom $\mu\gg 1$. I grensen der $\mu\rightarrow\infty$ blir analogien eksakt fordi permittiviteten er uendelig mye større inni kretsen enn utenfor (på samme måte som konduktiviteten i en leder er uendelig mye større enn konduktiviteten til vakuum, som er 0). De analoge størrelsene er
\begin{enumerate}
	\item Den magnetiske flukstettheten $\vB$ i en magnetisk krets tilsvarer den elektriske strømtettheten $\vJ$ i en elektrisk krets
	\item Den magnetiske fluksen $\Phi_S=\iint_S\vB\cdot\dS$ i en magnetisk krets tilsvarer den elektriske strømmen $I=\iint_S\vJ\cdot\dS$ i en elektrisk krets
	\item Loven om at $\oiint_S\vB\cdot\dS=0$ i en magnetisk krets tilsvarer loven om at $\oiint_S\vJ\cdot\dS=0$ i en elektrisk krets (men: den første gjelder alltid, den andre gjelder kun i den statiske grensen!).
	\item Ytre $\vH$-felt driver en magnetisk flukstetthet ($\vB=\mu\vH$) på samme måte som et ytre $\vE$-felt driver en elektrisk strøm ($\vJ=\sigma\vE$).
	\item Permittiviteten $\mu$ i en magnetisk krets tilsvarer konduktiviteten $\sigma$ i en elektrisk krets. 
\end{enumerate}
Flere analogier vil bli synlige i siste kapittel, når vi lar feltene variere med tiden.

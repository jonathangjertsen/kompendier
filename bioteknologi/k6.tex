\ctitle{Miljøbioteknologi}

\paragraph{Dette kapittelet} begynner med en lengre innføring i forurensning av ferskvann og hvordan vi kan forhindre det. Så går vi nærmere på hvordan vi kan bruke mikroorganismer som mer miljøvennlige energikilder og kjemikalieprodusenter.

\paragraph{Avløpsvann}\index{avløpsvann} Det er litt terminologi i forbindelse med avløpsvann:
\begin{enumerate}[noitemsep,nolistsep]
	\item \emph{5-dags Biochemical Oxygen Demand} (\ix{5-day BOD}): mengden \ce{O2} (aq) som metaboliseres av mikrobene i en vannprøve i løpet av 5 dager for å fullstendig bryte ned de organiske forurensningene i prøven.
	\item \idx{Population equivalent} (PE): den gjennomsnittlige organiske forurensningen, altså 60 gram 5-dags BOD.
\end{enumerate}
Siden løseligheten til \ce{O2} i vann er 10mg \ce{O2} pr. liter vann, trenger man 6000 liter rent vann for å fjerne forurensningen som én gjennomsnittlig bybeboer produserer i løpet av 5 dager. 

\paragraph{Konsekvensene av urenset avløpsvann} som slippes ut i innsjøer og elver, er at 
\begin{itemize}[nolistsep,noitemsep]
	\item aerobe mikroorganismer bryter ned organisk materiale og bruker samtidig opp tilgjengelig oksygen,
	\item fisk og de fleste andre oksygenavhengige organismer dør av oksygenmangel, og til slutt
	\item anaerobe mikroorganismer tar over og produserer giftig ammoniakk (\ce{NH3}) og hydrogensulfid (\ce{H2S}) som dreper de resterende vannlevende organismene.
\end{itemize}
Dermed ender man opp med at elvene blir til illeluktende kloakk.

\paragraph{Aerob vannrensning}\index{aerob vannrensning} Et vanlig kloakkrenseanlegg bruker aerobe mikroorganismer til å bryte ned forurensninger i vannet \emph{før} det slippes ut i ferskvannskildene. Dette krever en del plass.

\paragraph{Trickling filters}\index{trickling filters} En annen metode for å rense avløpsvann. Dette er tanker fylt med porøst materiale med et stort indre overflateareal der mikroorganismer kan sette seg. Dette store overflatarealet gjør at man sparer en del plass. 

\paragraph{Aktivert slam}\index{aktivert slam} En tredje metode for å rense avløpsvann. Her har man latt aerobiske bakterier, fungus og andre mikroorganismer danne store flak som andre mikrober kan sette seg på. I tillegg rører man i vannet for å stadig tilføre oksygen fra luften. Vannrensning med ``aktivert slam'' består av følgende trinn (se gjerne på figur på slide):
\begin{enumerate}[nolistsep,noitemsep]
	\item Avløpsvann går inn i en oppbevaringstank.
	\item Større objekter fjernes med mekanisk filter.
	\item I et ``gruskammer'': væskeflyt justeres, og større partikler som grus, sand og glass fjernes.
	\item I en sedimenteringstank: flakene av mikroorganismer, samt den løselige komponenten av avløpsvannet, separeres fra uløselig bunnfall (sedimenter).
	\item Flakene og den løselige komponenten av avløpsvannet brytes ned under aerobiske forhold for å lage ``aktivert slam'' (som beskrevet i starten av avsnittet).
	\item Rent vann separeres fra slammet. Deler av slammet resirkuleres.
	\item Slammet sendes til et ``fordøyelsestårn'' der det produseres biogass.
	\item Rent vann separeres fra restene av slam fra fordøyelsestårnet. Det siste slammet kan ende opp i spesialiserte dynger, der lukt kan bli et problem.
	\item Drikkbart vann behandles med klor eller ozon for å drepe bakterier, før vannet sendes ut i elva.
\end{enumerate}

\paragraph{Eutrofiering}\index{eutrofiering} Avrenning av fosfor og nitrogen (i gjødsel) fra landbruk, med påfølgende opphopning av slike næringsstoffer i miljøet. Dette fører til ukontrollert oppblomstring av alger, som igjen degraderes av aerobiske mikroorganismer som bruker opp det som er av tilgjengelig oksygen. Til slutt ender man opp i samme situasjon som man får fra å slippe ut urenset avløpsvann i ferskvannsområdene.

\paragraph{Fjerning av nitrogen}\index{fjerning av nitrogen}\index{kombinasjon av kjemisk syntese og bioteknologi} Først bruker man \idx{Nitrosomonas} til å oksidere ammoniakk til nitritt (\ce{NO2^-}). Deretter bruker man \idx{Nitrobacter} til å oksidere nitritt til nitrat (\ce{NO3^-}). Til slutt reduseres nitratet til nitrogengass\footnote{Slides strider med Renneberg og internett: reduksjonen av nitrat gjøres ikke biologisk.}.

\paragraph{Fjerning av fosfor}\index{fjerning av fosfor} Det finnes to metoder: fosforet kan enten tas opp av bakterier (som man så lagrer som ``biosolids'' som brukes som gjødsel), eller det kan presipiteres ut kjemisk med jern- eller aluminiumsalter.

\paragraph{Biogass}\index{biogass}\index{arkebakterier}\index{metanogenese}\index{metan} Metanogene arkebakterier kan konvertere organisk materiale til metan, som kan brukes som en energikilde. Dette skjer i en anaerob prosess:
\begin{enumerate}[noitemsep,nolistsep]
	\item Hydrolytisk fase: ekstracellulære enzymer bryter ned store organiske molekyler til enklere bestanddeler som monosakkarider, aminosyrer, glyserol og fettsyrer.
	\item Acidogen fase: de enkle bestanddelene brytes ned til \ce{H2}, \ce{CO2}, eddiksyre og andre organiske syrer og alkoholer.
	\item Metanogenese: Hydrogen, eddiksyre og karbondioksid reagerer med hverandre for å danne vann og metan i to reaksjoner: \ce{CH3COOH -> CH4 + CO2} og \ce{CO2 + 4 H2 -> CH4 + 2 H2O}.
\end{enumerate}
Biogass er en fin energikilde dersom det eneste andre alternativet er å brenne ved eller annen biomasse. Biogass kan dannes i små reaktorer som fylles med ekskrement fra mennesker og dyr.

Men: da man prøvde dette med \idx{Gobarprosjektet} i India, ble prosjektet forringet av at det kun er rike bønder som har råd til de rustfri ståltankene man trenger og har nok avfallsstoffer til å fylle tankene. Det endte opp med at rike bønder kjøpte opp kumøkk i store kvanta og dermed brukte opp biomassen som de fattige ellers ville brent for varme.

Det går bedre med biogass i Kina, der det kjøres mange hundre tusen bioreaktorer, ofte som kommunale prosjekter.

I den industrialiserte verden kan bioreaktorer hjelpe til med å løse avfallsproblemene til industrielle gårdsbruk. På grunn av begrenset mengde biomasse kan ikke biogass dekke stort mer enn 1-5\% av energibehovet.

Husholdningsavfall som ellers ville havnet på avfallsdynger er en annen potensiell kilde til biogass - det produseres allerede så mye metan på avfallsdynger at det kan være farlig, så det kan være gunstig å dekke dem opp og ekstrahere metan derfra.

\paragraph{Naturlig forekommende metan}\index{metan} Mikroorganismer som enten lever fritt eller i fordøyelsessystemene til større organismer, produserer totalt mellom 500 millioner til 1 milliard tonn metan i året, omtrent det samme som det som ekstraheres fra naturgassfelt. Eksempler er
\begin{itemize}[nolistsep,noitemsep]
	\item I myrer og sumper kommer det en del metan fra bakterier som dekomponerer organisk materiale i fravær av oksygen.
	\item Mikroorganismer i magen til en ku produserer 100-200 liter metan.
	\item Termitter er de største produsentene av metan: en maur produserer 0.5 milligram metan om dagen, som blir til en årlig produksjon på 150 millioner tonn.
\end{itemize}

\paragraph{Bioetanol}\index{bioetanol} I Brazil finnes det en del bioetanoldrevne biler som går på hydrert etanol (93\% etanol/7\% vann). Dette ble satt i gang på grunn av oljekrisene i 1973/4 og 1979. Prosjektet heter \emph{Proalcool} og var en både politisk og økonomisk motivert affære som har vist seg å være kontroversiell fordi
\begin{itemize}[noitemsep,nolistsep]
	\item bioetanolproduksjon konkurrerer med matproduksjon: én bil krever like mye råmateriale som mat for 30 mennesker.
	\item produksjonen har en alvorlig effekt på miljøet: hver liter etanol som produseres danner 12-15 liter sukkerrester og 100 liter skittent vann, som av økonomiske grunner slippes ut i elver uten å renses. Bioetanolproduksjonen fører også til erosjon av jordsmonnet.
\end{itemize}
Problemet med vannforurensning er på bedringens vei: etter at forurenset vann fra bioetanolindustrien forurenset drikkevannforsyningen til hele byer, beordret den brazilianske regjeringen alkoholindustrien til å holde vannet i en lukket syklus og til å bruke sukkerrestene som gjødsel.

\paragraph{Oljedegraderende bakterier}\index{oljedegraderende bakterier} Etter å ha oppdaget at det finnes bakterier som kan degradere plantegiften Agent Orange, klarte Ananda Chakrabarty å produsere oljespisende bakterier ved å kombinere plasmider fra fire \idx{Pseudomonias}-stammer som hver for seg var i stand til å degradere enten kamfer, oktan, xylol eller naftalen. Plasmidene ble kombinert til superplasmider som ble reintrodusert i bakterier slik at de ble i stand til å degradere alle de fire stoffene. Disse organismene var ment til å raskt spise opp oljeutslipp og så bli spist opp av organismer høyere opp i næringskjeden. Organismene er de første genmodifiserte organismene som har blitt patentert i USA.

Organismene har riktignok aldri blitt brukt, på grunn av restriksjoner på bruk av genmodifiserte organismer. I stedet gror man konvensjonelt fremavlede bakterier for å degradere olje på oljedekte steiner, som er det eneste som er lovlig å gjøre i dag.

\paragraph{Sukker og alkohol fra trevirke} Den ideelle råvaren for å lage glukose og alkohol er stivelse, et polysakkarid som brukes som energilager i planter. Stivelse er et viktig næringsstoff og er derfor kontroversielt å bruke industrielt.

Den nest beste råvaren er \ix{lignocellulose} fra trevirke, som ikke kan brukes som en matressurs, men er et råmateriale i papirindustrien. Lignocellulose er en 4:3:2-blanding av cellulose, hemicellulose og lignin, eller kan hentes fra levende planter, og som biprodukt fra jordbruk og trevirke (sagflis).

Noen problemer med lignocellulose som sukker- og alkoholkilde, er:
\begin{itemize}[nolistsep,noitemsep]
	\item Cellulose er krystallint, uløselig i vann og vanskelig nedbrytbart for de fleste organismer. Men treormer, termitter og visse sopper kan degradere cellulose med cellulaser.
	\item Degraderingsproduktene til lignin er giftige, slik at lignocellulose må syrebehandles før enzymdrevet nedbrytning.
	\item Cellulaser hemmes av sine egne produkter (glukose og cellobiose), og er uansett temmelig treige.
\end{itemize}

\index{hvitråte}Hvitråte-sopp kan degradere 60 til 70 prosent av ligninet til karbondioksid, vann og hvit cellulose. Dette gjør de med ekstracellulære peroksidaser som bryter bindingene mellom fenolene i lignin. Hvor hvitråte får hydrogenperoksiden som trengs for å drive denne reaksjonen fra, er et lite mysterium, men man lurer på om ikke det kan komme fra ekstracellulære oksidaser som glukoseoksidase, som lager \ce{H2O2} som biprodukt ved oksidering av glukose.

Mulige strategier for å forbedre prosessen er å
\begin{itemize}[nolistsep,noitemsep]
	\item fremtvinge mutasjoner som fjerner negativ feedback i cellulaser
	\item klone cellulaser fra fungus til mer økonomiske bakterier
	\item gi bakterier evnen til å bryte ned 5-ringede sukkerarter
	\item bruke mikrober som metaboliserer lignocellulose direkte for å danne etanol og organiske syrer
	\item lage nye enzymer som åpner opp krystallstrukturen til cellulose
\end{itemize}

\paragraph{Eksempler på kjemikalier fra biomasse}\index{kjemikalier fra biomasse} De 100 mest brukte kjemikaliene i industrien utgjør omtrent 99\% av massen av alle kjemikalier som brukes, og de 5 mest brukte kjemikaliene (eten, propen, benzen, toluen og xylen) utgjør omtrent 75\% av alle kjemikalier som brukes. Kjemikaliene kommer for det meste fra petrolium og naturgass, men alternativt kan de fleste kjemikaliene utvinnes fra bioteknologi:
\begin{itemize}[noitemsep,nolistsep]
	\item Etanol:\index{etanol} brukes som løsemiddel, antifrys, og som utgangsstoff for syntese av mange andre organiske forbindelser. Det kan som kjent utvinnes med fermentering, og destillasjonsprosessen kan i dag hjelpes på vei med termofile og etanoltolerante mikroorganismer.
	\item Eddiksyre:\index{eddiksyre} produseres fra oksidering av etanol av \idx{Acetobakter}, men høykonsentrert eddiksyre produseres ved kjemisk karbonylering av metanol. Et miljøvennlig bruksområde for eddiksyre er dannelse av kalsiummagnesiumacetat til salting av veier. I motsetning til natriumklorid fører det ikke til korrosjon i biler eller til tredød ved å konkurrere med næringsstoffene til planter.
	\item n-butanol:\index{n-butanol} viktig organisk løsemiddel. \idx{Clostridium acetobutylicum} produserer n-butanol med aceton som biprodukt. n-butanol er giftig for bakterier, men det problemet løses ved å bruke immobiliserte mikroorganismer.
	\item Glyserol:\index{glyserol} løsemiddel og lubrikant. Kan produseres med de samme mikrobene som man bruker for å produsere etanol, ved å legge til natriumsulfitt som binder seg til et viktig mellomprodukt i biologisk etanolsyntese.
	\item Sitronsyre:\index{sitronsyre} konserveringsmiddel, smakstilsetning og vaskemiddel. Produseres av \idx{Aspergillus}-soppen. 
	\item Melkesyre:\index{melkesyre} syreregulerende middel og konserveringsmiddel. Produseres av \idx{Lactobacillus}. Kan dehydreres til den sykliske laktonen laktid, som videre kan polymeriseres til den bionedgraderbare polyesteren polylaktat.
	\item Glukonsyre:\index{glukonsyre} lages av \idx{Aspergillus} niger fra glukose via glukoseoksidase. Kan brukes som biosensor for å måle glukoseinnhold. Binder metallioner og forhindrer kalsiumflekker på glass.
\end{itemize}

\paragraph{Er produksjon av industrielle kjemikalier fra fornybare kilder realistisk?} Produkter med høy verdi som skal produseres i lite volum, for eksempel aminosyrer, er det ofte bedre å få tak i bioteknologisk. For kjemikalier som skal produseres i store volumer er det gjerne billigere å bruke fossile kilder - bruk av bioteknologiske løsninger avhenger da i stor grad av oljeprisen og økonomisk lønnsomme bioprosesser. 

\paragraph{Bakterier i gruvedrift}\index{bioleaching} Levende organismer kan brukes til å ekstrahere metaller fra årer i en prosess som kalles ``bioleaching''. Omtrent 25\% av kopperproduksjonen, 10\% av gullproduksjonen og 3\% av kobolt- og nikkelproduksjonen bruker levende organismer på denne måten.

Mikrobene som brukes i kopperproduksjon er svovelbakteriene \idx{Acidithiobacillus} \emph{ferrooxidans}, som oksiderer Fe(II) til Fe(III) og angriper løselig svovel og uløselige sulfider for å konvertere dem til sulfater, og \emph{Acidithiobacillus thiooxidans}, som vokser på svovel og løselige svovelforbindelser.

I ``direct leaching'' får bakteriene energi ved å overføre elektroner fra jern og svovel til oksygen på cellemembranen. De oksiderte produktene, som er mer løselige, kan dermed lettere hentes ut. I ``indirect leaching'' oksideres Fe(II) til Fe(III), som i seg selv kan oksidere andre metallioner til lettere løselige former. I praksis er det noe overlapp mellom disse to prosessene.

\paragraph{MEOR}\index{MEOR} står for ``Microbially Enhanced Oil Recovery'', og er en betegnelse for tertiære oljeproduksjonsmetoder som brukes til å hente ut den siste oljen man ikke får tak i med primære og sekundære metoder. Én slik metode er å injisere bakterier som produserer gass og dermed øker trykket i reservene. 

En annen metode er å injisere mikroorganismer som lager \ix{biosurfaktanter} (bio-Zalo), som bryter opp oljen i vannløselige dråper som lettere kan hentes ut fra porer i stein. En utfordring med slike løsninger er de ekstreme forholdene ved oljereservoarene: det er lite næringsstoffer og oksygen, høy saltkonsentrasjon, høyt trykk og høy temperatur, så kun ekstremofile mikroorganismer kan overleve der.

Oljeprodusenter forsker også på bruken av biopolymerer som \ix{Xanthan}, som produseres av plantepatogenet \idx{Xanthomonas campestris}. Xanthan er et fortykkende stoff som gjør vann mer tyktflytende. Etter at biosurfaktanter har blitt pumpet inn i steinen, legger man til xantanfylt vann som utøver et trykk på oljen og presser den ut. Xanthan er fortsatt for dyrt til å være lønnsomt til slik bruk.

\paragraph{Bioplast}\index{bioplast} Plast som utledes fra fornybare kilder til biomasse, for eksempel vegetabilsk olje, stivelse og mikroorganismer. Mange, men ikke alle typer bioplast er laget for å kunne brytes ned biologisk i enten anaerobiske eller aerobiske miljøer. 

Eksempel på bioplast:
\begin{itemize}[nolistsep,noitemsep]
	\item \index{pullulan}\emph{Pullulan} er et cellofan-lignende polysakkarid som ikke kan fordøyes av mennesker eller degraderes av stivelsesnedbrytende amylaser, og brukes derfor som en kalorifattig tilsetning til mat for å øke viskositeten. Xanthan, som ble nevnt i forbindelse med oljeutvinning, brukes til det samme. Pullulan er også en bra lufttett matfolie som løser seg opp i varmt vann og kan nedbrytes mikrobielt når det er vått. Pullulan lages med \emph{Pullularia pullulans}-soppen.
	\item Polyhydroksybutyrat (PHB), med navnet \idx{Biopol}, har de samme egenskapene som petroleumsproduktet polypropylen og produseres av bakterier som energilager (i så stor grad at bakteriene hovedsakelig består av plast). PHB fra \emph{Alcaligenes eutrophus}-bakterien er en høykrystallin bioplast med smeltepunkt nær 180 celsiusgrader. Det brytes ned i en slik hastighet at det kan brukes i kapsler for medisin som skal slippes ut i kroppen over lengre tid, eller i sting (slik at de etter hvert fjerner seg av seg selv).
	\item Man forsøker å produsere edderkoppsilke med recombinant \idx{E. coli} under navnet \ix{Biosteel}.
	\item \index{polylaktat}Polylaktat, under navnet NatureWorks PLA, dannes ved naturlig fermentering av kornstivelse og kan brukes til å lage miljøvennlige konvolutter, dørmatter, CD-er og lignende. 
\end{itemize}

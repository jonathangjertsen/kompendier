\ctitle{Analytisk bioteknologi}

\paragraph{Dette kapittelet} forklarer bioteknologisk måling og analyse, herunder: hvordan vi kan bruke enzymer og organismer til å teste nivåer av spesifikke molekyler; forskjellige metoder for å analysere DNA; og genomikk, en disiplin innen genetikken der man analyserer funksjonen og strukturen til hele genomer.

\cstitle{Biologiske tester}\index{biologisk test}

\paragraph{Enzymbaserte biologiske tester} er basert på reaksjoner som katalyseres av spesifikke enzymer. Testen innebærer at et enzym, som regel produsert av en rekombinant mikroorganisme, konverterer et spesifikt molekyl til et produkt som kan detekteres og måles. En slik test må være sensitiv nok til å detektere små mengder av spesifikke molekyler. 

Et eksempel på en slik test er å bruke \ix{glukoseoksidase} til å teste for diabetes. Glukoseoksidase tar elektroner fra glukose og bruker dem til å redusere oksygen til hydrogenperoksid. Denne reaksjonen kjøres samtidig med en peroksidasekatalysert reaksjon, der en indikator skifter farge i reaksjon med hydrogenperoksid. Dermed kan aktiviteten til glukoseoksidase, som er proporsjonal med glukosenivået, måles med kolorimetri. Testen brukes for å måle glukosenivå i blod og kroppsvæsker. Responsen til testen kalibreres med prøver med konstant mengde enzym og forskjellige mengder glukose.

\paragraph{Biosensor}\index{biosensor} Enzym- og mikroorganismebaserte innretninger som brukes til å detektere spesifikke molekyler i en prøve. Disse kombineres som regel med elektronikk for å tolke og kvantifisere et signal.

Et eksempel på en engangs-biosensor er glukoseoksidase som immobiliseres \index{immobilisering} på en chip. Glukose i prøven binder FAD på det aktive setet i glukoseoksidase, og elektroner trekkes fra glukose gjennom \ix{FAD}. Disse elektronene tas opp av \ix{ferrocen} (en artig organisk forbindelse der et jernatom er klemt mellom to syklopentadienylringer i en sandwich-konfigurasjon), som diffunderer til chip-en og donerer elektronene sine der. Den resulterende elektriske strømmen, som er proporsjonal med glukosekonsentrasjonen, måles.

Et eksempel på en \emph{gjenbrukbar} biosensor er glukoseoksidase som immobiliseres på en membran som dekkes med en gel - se figur i Renneberg for oppsettet. Kun oksygen og andre små molekyler kan slippe gjennom membranen til siden med glukoseoksidase. Glukoseoksidase konverterer glukose til glukonolakton og reduserer oksygen til hydrogenperoksid, som vanlig. Hydrogenperoksid donerer så elektronene sine til en elektrode og genererer en elektrisk strøm som kan måles og er proporsjonal med glukosekonsentrasjonen.

\paragraph{Organismebasert biosensor} Også levende immobiliserte mikrober kan brukes i biosensorer. Vanligvis bruker man gjærceller i en polymer-gel på en oksygenelektrode. Det vanligste bruksområdet er måling av bionedgraderbare forurensninger. Som beskrevet i kapittel 6 vil mikrober konsumere en mengde oksygen som er proporsjonal med mengden degraderbare organiske forbindelser i prøven. Derfor kan man bruke en slik test til å måle forurensningsnivåer indirekte ved å måle oksygennivået i en prøve der man har tilsatt gjærceller. Immobilisering \index{immobilisering} av gjærcellene gjør at man kan måle testen på 5 minutter i stedet for å gjøre en 5-dags BOD-test.

\paragraph{Immunologisk testing}\index{immunologisk testing}\index{antistoff} Antistoffer gjør det mulig å utføre diverse tester:
\begin{itemize}[nolistsep,noitemsep]
	\item Graviditetstest:\index{graviditetstest} graviditet medfører en dramatisk økning i hormonet hCG (``human chorionic gonadotropin''). \ix{hCG} kan detekteres i urin på en papirstripe med monoklonale antistoffer for hCG. På den ene siden av papiret binder antistoffet seg til en bestemt epitop på hCG, og det resulterende komplekset beveger seg, på grunn av kapillærkrefter, til motsatt side av papiret. I midten av papiret er det en stripe med immobiliserte antistoffer som binder seg til en annen epitop på hCG. Det resulterende ``sandwich-komplekset'' av hCG og to antistoffer er synlig som en stripe på papiret. Hvis det ikke er hCG til stede vil ikke de immobiliserte antistoffene kunne binde seg til resten av antistoffene, som blir dratt videre av kapillærkreftene helt til den andre siden av papiret. Det er alltid én kontrollstripe ved siden av for å bekrefte at testen fungerer, så to striper indikerer graviditet.
	\item Virusinfeksjon-tester: virusinfeksjoner kan detekteres med ELISA-testen som ble beskrevet i kapittel 5.
	\item Hjerteinfarkt-tester:\index{hjerteinfarkt} Døende hjerteceller slipper ut spesifikke proteiner som kreatinkinase og troponiner, som er pålitelige indikasjoner på hjerteinfarkt og kan måles direkte - men disse kan tidligst detekteres når hjerteinfarktet har pågått i en time. Med monoklonale antistoffer kan man måle andre proteiner som slippes ut tidligere, men som er vanskeligere å måle, for eksempel det lille proteinet h-FABP. Testen fungerer på en lignende måte som graviditetstesten.
	\item Point-of-Care-tester (POC): ved å måle forskjellige typer lipider i blodet kan man finne ut av risikoen for hjerteinfarkt eller hjerneslag.
\end{itemize}

\begin{centering}\subsection{DNA-analyse}\end{centering}

\paragraph{Gel-elektroforese}\index{gel-elektroforese} DNA kan identifiseres ved å bruke de samme \index{restriksjons-endonuklease} restriksjons-endonukleasene som vi brukte for å lage rekombinante bakterier i kapittel 3. Siden enzymene kutter på svært spesifikke steder, vil man få en blanding DNA-fragmenter med forskjellig fordeling av størrelser avhengig av DNA-sekvensen. Dermed kan man, ved å separere DNA-fragmentene basert på størrelse, danne et ``genetisk fingeravtrykk'' som gjør det mulig å se om to DNA-prøver kommer fra det samme individet. Se figur i bok/internett for å forstå oppsettet. Dette er prinsippet som brukes i rettsmedisinsk DNA-analyse. 

Ved DNA-testing er intronene, \index{intron} de ``ubrukelige'' delene av DNA-et som ikke koder for gener, nyttige. Siden mutasjoner i intronene som regel ikke har noen effekt på organismen, vil mutasjoner i introner lettere videreføres fra en generasjon celler til den neste (i motsetning til mutasjoner i DNA som koder for gener, som sannsynligvis vil føre til at immunsystemet destruerer cellen). Denne høyere mutasjonsraten i introner gjør at de er kan være svært forskjellige fra individ til individ. Dermed kan de genetiske fingeravtrykkene til forskjellige individer, til og med eneggede tvillinger, være ganske forskjellige.

\paragraph{\ix{Southern blotting}} En dramatisk forbedring av gel-elektroforese som gjør det mulig å detektere spesifikke gensekvenser. Før elektroforese denatureres DNA-et (det deles opp i de to enkeltstrengene det består av) med natronlut. Etter elektroforese plasseres gelen på et nitrocellulosefilter, og det påføres trykk (for eksempel ved å putte papirtørkler oppå). Da vil kapillærkrefter føre til at DNA-strengene fester seg på filteret og binder seg der, slik at man ender opp med et bilde av DNA-et i gelen på filteret. 

Så hybridiserer man DNA-et ved hjelp av DNA-prober med sekvensene man ønsker å teste (som beskrevet i kapittel 3). Hvis man for eksempel har en fluorescerende probe, vil påfølgende røntgenstråleeksponering gjøre at det hybridiserte DNAet ``lyser opp''.

Grunnen til at man må gjøre alle greiene med nitrocellulosefilter før hybridisering, er at DNAet er lettere tilgengelig og mindre mobilt på filteret enn i gelen.

\paragraph{\ix{PCR}} står for ``Polymerase Chain Reaction'', og er en metode for å produsere mange kopier av et DNA-molekyl fra ett enkelt stykke DNA. Metoden bruker DNA polymerase, det samme enzymet som brukes for å kopiere DNA under celledeling i naturen. Metoden utføres slik: 
\begin{enumerate}[nolistsep,noitemsep]
	\item Løsningen med DNA varmes opp til 92-94 grader. Dette fører til at DNA-molekylet spaltes opp i sine to enkeltstrenger.
	\item Løsningen kjøles ned til 42-55 grader, og ``primers'' legges til. Dette er små oligonukleotuder som binder seg på DNA. Primers er nødvendig fordi de gir DNA polymerase et startpunkt - DNA polymerase trenger et startpunkt der det finnes dobbeltstrenget DNA.
	\item Nukleotider legges til i overskudd, og DNA polymerase syntetiserer datter-DNA på hver av de to enkeltstrengene fra det opprinnelige DNA-molekylet.
	\item Prosessen repeteres: løsningen varmes opp igjen, og hver av de to DNA-molekylene gir opphav til to nye molekyler. I løpet av noen minutter har man massevis av kopier av det ene DNA-molekylet man begynte med.
\end{enumerate}
DNA polymerasen som brukes, er ikke fra mennesker, men en genmodifisert variant av det tilsvarende enzymet fra den ekstremofile bakterien \index{ekstremofile organismer} \idx{Thermus aquaticus}. Dette enzymet tåler det første steget der løsningen varmes opp til 92-94 grader (i motsetning til menneskelig polymerase, som blir ødelagt av slike temperaturer). Før man fant på den lure løsningen om å bruke ekstremofile enzymer, måtte man legge til nytt DNA polymerase en gang per oppvarmingssyklus.

Se boks 10.5 i tillegg til seksjonen om PCR i Renneberg!

\paragraph{DNA-sekvensering} \index{DNA-sekvensering} er metodene man har for å finne den eksakte nukleotidsekvensen i DNA. Dette er nyttig for å utlede aminosyresekvensene i proteiner, finne den eksakte sekvensen til et gen, finne regulerende elementer som promotere, finne forskjeller i gener og å finne mutasjoner.

\paragraph{\ix{Sangers kjede-termineringsmetode}} er den vanligste metoden for å sekvensere DNA, og ble funnet opp av \ix{Frederick Sanger} i 1977. 
\begin{enumerate}[noitemsep,nolistsep]
	\item DNA denatureres, som beskrevet i Southern Blotting og PCR. En av de to strengene tas ut, denne kalles en ``template''-streng
	\item Template-strengen hybridiseres med en radioaktivt merket primer, for å tillate at nukleotider legges til senere.
	\item Man lager 4 separate løsninger. Til alle løsningene legger man til DNA-et som skal sekvenseres, som nå har en primer på seg.
	\item DNA polymerase legges til i de 4 løsningene.
	\item Frie nukleotider (dNTP, deoksyribonukleotider) legges til i de 4 løsningene. Alle de 4 nukleotidene i DNA (A, T, G, C) legges til i alle 4 løsningene.
	\item Modifiserte nukleotider (ddNTP, dideoksyribonukleotider) legges til i de 4 løsningene. \emph{Kun én type modifisert nukleotid legges til i hver løsning}. Vi kan kalle de modifiserte nukleotidene A*, T*, G* og C*. ddNTP er som dNTP, men en OH-gruppe er erstattet med et H-atom. Dette gjør at DNA polymerase må stoppe etter at det har slengt på et modifisert nukleotid i kjeden, fordi det ikke er noe OH-gruppe som den kan hekte neste nukleotid på.
	\item Prøver tas ut av alle løsningene, og DNA-fragmentene i de 4 løsningene sammenlignes basert på størrelse ved hjelp av gel-elektroforese.
\end{enumerate}
Nå kan nukleotidsekvensen leses ut direkte fra gel-en! Hvis det nederste merket i gel-en er fra A*-løsningen, var den første basen i sekvensen (etter primer-seksjonen) den komplementære basen T. Det nest nederste merket gir informasjon om den nest første basen i sekvensen, og så videre.

Hvorfor fungerer det? Renneberg og slides er ikke veldig gode for å beskrive Sangers kjede-termineringsmetode og hvorfor det fungerer. Disse videoene forklarer det bedre: \url{https://www.youtube.com/watch?v=nudG0r9zL2M} og \url{https://www.youtube.com/watch?v=vK-HIMaitnE}. Den viser også hvorfor det er begrenset hvor store sekvenser som kan leses med én iterasjon av metoden (grensen er på ca. 400 basepar). Det er riktignok mulig å lese lengre sekvenser ved å gjennomføre metoden flere ganger og sette sammen resultatene fra hver runde som i et puslespill.

\paragraph{Automatisk DNA-sekvensering} Datamaskiner gjør det mulig å lese ut lengre DNA-sekvenser. Med den automatiske metoden putter man fluorescerende merkelapper på de modifiserte nukleotidene (ddNTP). Man har forskjellige merkelapper på hver av nukleotidene, så reaksjonen kan skje i én prøve i stedet for 4. Reaksjonen utføres på samme måte som i Sangers kjede-termineringsmetode, men produktene separeres med kappilær elektroforese i stedet for gel-elektroforese. Så eksiterer en laser de fluorescerende merkelappene, og de modifiserte nukleotidenes respons måles med en detektor. Basert på kombinasjonen av plassering i den kapillære tuben (som gir informasjon om reaksjonsproduktets størrelse, det vil si når polymerasereaksjonen terminerte) og signalet fra detektoren (som gir informasjon om hvilken modifisert nukleotid som forårsaket termineringen) kan man, med tilstrekklig lur software, finne DNA-sekvensen.

\paragraph{\ix{FISH}} står for ``fluorescence in situ hybridization'' og er en metode for å finne ut på hvilket kromosom et gen ligger, og om det finnes mange kopier av genet spredt utover forskjellige kromosomer. Genene spres ut på en glassplate, og man legger til en DNA-probe med det komplementære DNA-et til genet man skal undersøke, med en fluorescerende markør. Proben hybridiserer med de relevante genene, slik at de lyser opp under fluoerescensmikroskop. 

FISH kan brukes til å identifisere kreftceller, som har karakteristiske forvrengninger i DNA-et; blant annet er rekkefølgen av de 23 kromosomparrende mikset opp. Teknikken er spesielt effektiv når man bruker forskjellige markører med forskjellige farger (da kalles det ``\ix{multicolor banding}'').

FISH-tester kan også brukes til å studere strukturen til kromosomer eller finne kromosomfeil.

\cstitle{Genomikk}

\paragraph{Genom}\index{genom} En komplett kopi av den genetiske informasjonen til en organisme kalles organismens genom.

\paragraph{Genomikk}\index{genomikk} Disiplinen innenfore genetikk som innebærer å kartlegge, sekvensere og analysere funksjonene til hele genomer. Analysen av genomets funksjon kalles funksjonell genomikk.

\paragraph{The Human Genome Project}\index{Human Genome Project} Et prosjekt som foregikk fra 1990 til 2005, med det formål å kartlegge hele det menneskelige genom - alle de 3.4 milliarder baseparene -, konstruere et detaljert fysisk kart over genomet og å bestemme nukleotidsekvensen til alle de 23 menneskelige kromosomene.

Presis kunnskap om det menneskelige genomet er et viktig verktøy for å behandle og forebygge genetiske og genetisk påvirkede sykdommer. Blant disse har man de monogenetiske sykdommene som forårsakes av en feil i ett enkelt gen (som Huntington's, fenylketonuri og cystisk fibrose), men også sykdommer som kreft og astma har genetiske komponenter. 

\paragraph{Genomisk kartlegging}\index{genomisk kartlegging} Siden genomet til en organisme består av milliarder av basepar, men sekvenseringsteknologien som er beskrevet over er begrenset til sekvenser på 750 basepar, må genomet settes sammen som et puslespill. For å hjelpe til med puslespillet trengte man \emph{genomiske kart} som inneholder informasjon om bestemte jevnt fordelte ``markører''. Slike genomiske kart finnes i to typer - genetiske kart og fysiske kart.\footnote{Rennebergs beskrivelse av genomisk kartlegging oppleves av undertegnede som en ugjennomtrengelig ordsalat som overhodet ikke forklarer forskjellen mellom genetiske og fysiske kart, noe som kommer til uttrykk i de noenlunde intetsigende avsnittene jeg har skrevet om temaet. Artikkelen \emph{What type of genome maps are there?}, lenket til i Tonsill B, gir en mye mer forståelig forklaring av den prinsipielle forskjellen mellom genetiske og fysiske genomkart, men den er så annerledes fra det Renneberg sier at jeg lurer på om de to beskrivelsene engang er enige. Ifølge en slik beskrivelse handler genetiske kart om hvilke gener som ofte arves sammen (som gir et mål på ``genetisk avstand''), mens fysiske kart er basert på den faktiske fysiske avstanden mellom to sekvenser, målt i antall basepar fra det ene til det andre. Den eksamens-engstelige leser kan beroliges med at det trolig bare er nødvendig med overordnet kunnskap om genomikk og The Human Genome Project.}

\paragraph{Genetiske kart}\index{genetisk kart} I genetiske kort er markørene korte stykker der sekvensen CACACA repeteres. Slike sekvenser kalles STRs \index{short tandem repeats} for ``short tandem repeats''. Slikt DNA finnes på alle kromosomer, og antallet repetisjoner i en STR kan man finne ut ved å bruke komplementært DNA med forskjellige lengder (f.eks Primer-CACACA-Primer og Primer-CACACACACACA-Primer) og kjøre det gjennom PCR og gel-elektroforese. Hvor langt det hybridiserte DNAet kommer i gel-en avhenger av hvor lang STR-en er.

\paragraph{Fysiske kart}\index{fysisk kart} viser posisjonen til bestemte DNA-sekvenser på DNA-molekylet. Disse sekvensene finner man ved å bruke revers-transkriptase \index{revers-transkripsjon}\index{voksent mRNA} på voksent mRNA for å lage DNA. Dette DNA-et brukes som prober i FISH eller andre teknikker for å finne posisjonen til genet på kromosomet.

\paragraph{Avansert genomanalyse} Noen aspekter ved avansert analyse av det menneskelige genomet er:
\begin{itemize}[nolistsep,noitemsep]
	\item \ix{DNA-sekvensering}\index{shotgun-metoden}\index{contig-metoden}: her finnes det to metoder. ``Contig''-metoden, som ble brukt i Humane Genome Project, er: lag kloner av genomet, kutt opp med restriksjons-endonukleaser, introduser de resulterende fragmentene i plasmidet til bakterier, sekvenser hver av segmentene med Sangers metode eller tilsvarende, og prøv å sette det hele sammen basert på genetiske markører. En annen metode er ``Shotgun''-metoden til Craig Venter sitt firma Celera Genomics, som man tidligere hadde brukt for å sekvensere genomet til \idx{Haemophilus influenzae}-bakterien. Shotgun-metoden er å tilfeldig fragmentere genomet med ultralyd, sekvensere fragmentene, og prøve å se etter overlapp. Dette fungerer bare hvis man finner overlapp mellom alle sekvenser. Det går fint med prokaryote celler som inneholder få repeterende sekvenser, men eukaryote celler har mange STR-er som gjør at man kan få falske positive resultater. Venter måtte utvikle kompliserte algoritmer for å løse problemet, og måtte til slutt støtte seg på resultatene fra Contiq-metoden til Human Genome Project.
	\item Bioinformatikk:\index{bioinformatikk} utviklingen av softwarealgoritmer som finner, setter sammen og analyserer DNA-sekvenser.
	\item Transkriptomikk:\index{transkriptomikk} studiet av genuttrykk, et verktøy for å studere endringer i genaktivitet og hvordan de kan fremkalle sykdommer. Her studerer man \ix{voksent mRNA}, hentet fra både friskt og sykt vev. Slikt voksent mRNA inneholder kun genetisk informasjon for uttrykte gener, så ved å sammenligne genuttrykket i friskt og sykt vev kan man vise hvilke endringer i genuttrykket som forårsaker sykdom. Til dette brukes DNA-chips - matriser med tusener av kjente genfragmenter. Reverstranskriptase brukes for å lage DNA (med fluorescerende eller radioaktive markører) som kan hybridisere med det komplementære DNAet på chip-en, slik at man kan se hvilke gener som er uttrykt.
	\item Proteomikk:\index{proteomikk} identifisering av alle proteinene i en celle; hvordan uttrykkes proteiner, og hvordan fungerer og interagerer de? Proteomikk består av å isolere enkeltproteiner, og så gjøre detaljert strukturell analyse av proteinene.
\end{itemize}

\paragraph{Farmakogenomikk}\index{farmakogenomikk}\index{personalisert medisin} Personalisert medisin basert på individets genetikk. Forskjellige pasienter har forskjellige gener som interagerer med medisin, og derfor vil medisin fungere forskjellig for forskjellige pasienter. Dette medfører at svært mange pasienter ikke har noen nytte av medisinene sine. Den genetiske profilen til pasienten blir dermed et enormt nyttig verktøy for å bestemme hvilken medisin pasienten skal ta, og gjør at man kan ta medisin som er mer effektiv og har færre bivirkninger. I tillegg er det et nyttig verktøy for diagnostisk risikoanalyse.

\ctitle{Kloning og embryoer}

\paragraph{Dette kapittelet} handler om kloning av høyere ordens organismer. Mot slutten, litt diskusjon rundt bruk av embryoer til andre ting enn kloning.

\paragraph{Kloning}\index{kloning} Når en organisme får avkom som er genetisk helt identisk. Eksempler på naturlig kloning er den aseksuelle reproduksjonen til bakterier, samt noen insekter og planter.

\paragraph{Somatisk celle}\index{somatisk celle} En celle som er en del av kroppen til en organisme, det vil si alle andre celler enn kjønnsceller og stamceller (kapittel 9).

\paragraph{Overføring av kjerne}\index{overføring av kjerne} En måte å lage kloner på, er å overføre kjernen fra en somatisk donorcelle til en eggcelle der man har fjernet kjernen. Eggcellen blir så til et embryo, som settes inn i en surrogatmor. Kloner som produseres slik er ikke egentlig genetisk identiske med organismen som donorcellen kom fra, siden klonen vil ha mitokondrielt DNA fra eggcellen.

De første eksperimentene med overføring av kjerne ble gjort med frosker og salamandre. Da man klarte å lage kloner med denne metoden, viste man at voksne somatiske celler
\begin{itemize}[nolistsep,noitemsep]
	\item inneholder all den genetiske informasjonen som trengs for å skape hele organismen
	\item har evnen til å aktivere fosterutvikling
	\item kan ``glemme'' spesifikasjonene sine og oppføre seg som et totipotent (kapittel 9), befruktet egg
\end{itemize}

\paragraph{Reproduktiv kloning} Innebærer at man, etter overførig av kjerne, lar den nye hybridcella vokse til et embryo som settes inn i en surrogatmor med det formål å lage en ny organisme.

\paragraph{Terapeutisk kloning} Innebærer at embryoet, i stedet for å implanteres i en surrogatmor, las vokse \emph{in vitro} til en blastocyst. Så høster man cellene fra innsiden av blastocysten og overfører dem til et medium som fremmer vekst av visse typer celler (for eksempel muskel-, benmarg- eller nerveceller) med det hensyn å produsere vev som kan brukes i transplantasjoner.

\paragraph{Sauen Dolly}\index{Dolly} Det første klonede pattedyret, født i 1996. Dolly ble laget med overføring av kjerne, der kjernen kom fra jurene til en voksen sau. Dolly var det ene vellykede forsøket ut av de 277 kloningsforsøkene som ble gjort, hvorav 29 forsøk nådde stadiet der man hadde et embryo som kunne overføres til en surrogatmor (men i alle andre tilfeller enn Dolly ble fostrene spontanabortert på et tidlig stadium). 

Et problem med voksne celler er at mye av genmaterialet er undertrykt (kapittel 4, ``Strategisk adaptasjon''). For å lage Dolly ble denne ``genblokkaden'' fjernet ved å sulte donorcellene i et næringsfattig medium. Sannsynligvis gjorde dette at DNA-innpakkingen ble revertert til sin opprinnelige tilstand.

\paragraph{Årsaker til begrenset levetid} Klonede dyr har en tendens til å dø tidligere enn andre (Dolly døde 6 år gammel av en lungesykdom som er typisk for eldre sauer).

Én mulig årsak er at cellene til kloner har kortere telomerer \index{telomer} (repeterende DNA-sekvens på endene av kromosomene som beskytter kromosomene fra skade, som blir kortere ved hver celledeling og dermed er en indikator for cellens alder) enn andre dyr. 

En annen mulig årsak er at DNA-skade som er usynlig i en voksen celle kan komme frem når genmaterialet skal brukes til å lage forskjellige celler (tenk for eksempel hva som skjer hvis man bruker en levercelle som ser helt fin ut, men som har skade i genene som er aktiverte i nerveceller, men ikke i leverceller).

En tredje mulig årsak er feil i interaksjonen mellom cellen man har fjernet cellekjernen fra og cellekjernen man har injisert inn i cella.

\paragraph{Kloning av utryddede dyr} Det gjøres forsøk på å klone mammut med somatiske mammutceller som har blitt bevart i den sibirske permafrosten. Hvis kjerner fra slike celler kan settes inn i eggcellene til en elefant, bør det i prinsippet være mulig å lage en ny \ix{mammut}. Forhåpentligvis prøver man ikke å lage dyrepark av det denne gangen.

\paragraph{Kloning av mennesker} På tross av påstander fra diverse konspirasjonsteoretikere og andre tullinger rundt omkring i verden, er det trolig ingen som har klart å overvinne de tekniske utfordringene med å klone mennesker. Siden det skjer ganske mange spontanaborter og misdannelser for hver vellykkede kloning, og siden kloning av mennesker har en tendens til å være ulovlig (blant annet i EU gjennom Lisboa-traktaten), er ikke klonede mennesker noe som forskes på i noen særlig grad.

\paragraph{Etikk rundt kloning} Det er tynt med diskusjon rundt de etiske konsekvensene av kloning i Renneberg. To argumenter som presenteres mot reproduktiv kloning av mennesker, er 
\begin{inparaenum}[(i)]
	\item at måten mennesker lages på blir en teknisk prosess, og 
	\item at man prøver å skape mennesker i sitt eget bilde.
\end{inparaenum}

Også kloning av andre dyr enn mennesker har etiske konsekvenser knyttet til dyrevelferd.

\paragraph{Lovgivning rundt kloning}\index{regulering i Norge} I Norge er forskning på befruktede egg og kloning av mennesker forbudt ved den norske bioteknologiloven siden 2004. Terapeutisk kloning, altså kloning av celler til bruk i medisin, er tillatt, men ofte bruker man heller stamceller til det samme formålet.

I USA er kjøtt og andre produkter fra klonede dyr FDA-godkjent uten at maten må merkes på noen spesiell måte, fra og med 2006. Dette fordi mat fra klonede organismer er identisk med mat fra organismene som ble klonet.

\paragraph{Preimplantasjonsdiagnostikk (PID)}\index{preimplantasjonsdiagnostikk} Prosedyrer innenfor medisin og genetikk som utføres på embryo før de settes inn i en livmor. Når embryoet er på 8-cellestadiet kan man fjerne 2 celler (en for analyse, en for backup av analysen) uten at embryoet tar skade. Denne embryoniske cellen kan så destrueres for å studere genmaterialet inni. Informasjon som man kan finne ved slik analyse inkluderer kjønn (siden man kan se forskjell på X- og Y-kromosomet fordi Y-kromosomet er mindre) og alvorlige kromosomfeil (fordi det er lett å finne et ektra eller manglende kromosom). I prinsippet skal det også være mulig å lese hele DNA-et, men med dagens teknologi ville det vært ekstremt dyrt. 

PID innebærer naturligvis et lass med etiske problemstillinger, fremst blant dem: er vi egentlig tjent med å kunne bruke slik teknologi til å kunstig velge fostere med spesielle egenskaper (som kjønn, og etter hvert som teknologien forbedres, andre genetiske egenskaper)? Det kan i ytterste konsekvens føre til selektiv abort og et sorteringssamfunn. I likhet med problemstillingene rundt kloning diskuteres ikke disse i alt for stor detalj i Renneberg.

\paragraph{Hvor mye forteller genmaterialet ditt om deg?} Det vet man ennå ikke helt sikkert, siden mange arvelige egenskaper ikke avhenger av bare ett gen, men flere gener og interaksjonene mellom dem. Men man vet også at det meste er avhengig av miljøet man vokser opp i: størrelse og helse avhenger av ernæring, intelligens er sterkt påvirket av de hva som skjer de tre første leveårene, og man kan være genetisk predisponert for mange sykdommer som bare kommer til syne hvis man lever usunt eller under stressede leveforhold.

\paragraph{Bruk av genetisk data} Det er umulig at en \ix{DNA-profil} kan gi en komplett beskrivelse av et menneske og deres personlighet, fordi de rundt 3 milliarder basepar i DNA-et umulig kan kode for de rundt 100 milliarder nevroner som hver er koblet til rundt 1000 andre nevroner. Siden en DNA-profil ikke er en personlighetsprofil, må det være strenge regler som forbyr arbeidsgivere, forsikringsselskaper og andre aktører å bruke genetiske data til andre formål enn det formål som gjorde at man samlet dataene i utgangspunktet.

Med tanke på personvern kan man ikke si så alt for mye om en person basert på genene deres (i hvert fall ikke i dag), men det kan bli et farlig verktøy i kombinasjon med annen informasjon som pasientjournaler, nettlesningslogger, telefonsamtaler og bankkontoaktivitet.

\ctitle{Grønn bioteknologi}

\paragraph{Dette kapittelet} handler om hvordan bioteknologiske løsninger kan øke matproduksjonen, redusere feilernæring og produsere medisiner som består av komplekse molekyler. ``Grønn'' er her i sammenheng med at vi snakker om planter, ikke at vi prøver å lage miljøvennlige løsninger (det var kapittel 6).

\cstitle{Mikrober}

\paragraph{Alger} Blant \ix{alger} har vi makroalger og mikroalger. Alger som er store nok til at man kan se dem med det blotte øye kalles makroalger, resten kalles mikroalger. Makroalger er mer økonomisk lønnsomme enn mikroalger. Blant makroalger har vi
\begin{itemize}[nolistsep,noitemsep]
	\item brunalger, for eksempel tang og tare. Brunalger er råvaren for å lage alginat, agar, carrageenan og L-glutamat. Brukes i salater, supper, nudler eller med kjøtt.
	\item rødalger, som har blitt kultivert i Japan siden middelalderen.
	\item grønnalger, som antakelig er grønne.
\end{itemize}

Blant mikroalger har vi blåalger som \idx{Spirulina}, som egentlig ikke er alger, men cyanobakterier. De ble spist av indianerstammer i Sør-Amerika og i Afrika, og tas si dag som kosttilskudd for å redusere kolesterol og rense blodet. 100 gram spirulina inneholder 70 gram protein, 20 gram sukker, 2 gram fiber, 2 gram fett, samt en del viktige vitaminer og mineralsalter. Spirulina kan gros i algefarmer - store, flate vannbasseng - men det er fortsatt dyrere enn soyaprotein.

\paragraph{Spiselige mikrober}\index{spiselige mikrober} Mange mikrober inneholder verdifulle proteiner, fettsyrer, sukkerarter og vitaminer, og kan brukes som matkilder for å redusere sult. Eksempler på slike mikrober er mikroalger som \idx{Spirulina}, samt gjær og bakterier. Som matkilde har mikrober den fordelen at de vokser svært fort: tiden det tar for en organisme å fordoble sin biomasse er 2 måneder for en kalv, 6 uker for en grisunge, 4 uker for en kylling, 2 uker for gress, 6 timer for mikroalger, 4 til 6 timer for gjær og 20 minutter til 2 timer for bakterier. Dessuten blir nesten alt fôret konvertert til spiselig protein, i motsetning til en ku som bare inneholder kjøtt tilsvarende 9\% av planteproteinet den spiser.

\paragraph{\ix{Single Cell Protein}} Produksjon av protein med mikroorganismer begynte under 1. verdenskrig, da man avlet store mengder gjær som proteinkilde i pølser og supper. Gjær var nyttig fordi gjær benytter seg av ellers ubrukelige sukkerløsninger, og kan konvertere sukker til nyttig protein. 

Senere har man opdaget lignende organismer som spiser andre ``ubrukelige'' stoffer i råolje og konverterer dem til nyttig protein:
\begin{itemize}[nolistsep,noitemsep]
	\item i Øst-Europa konsentrerte man seg om \idx{Candida}-gjærceller som spiste alkaner. Bruken av slike er begrenset, i frykt for at cellene lager kreftfremkallende biprodukter.
	\item i Vest-Europa fant man gjær og bakterier som gikk på metanol, som man tenkte å bruke som dyrefôr. Dette ble ikke noen suksess, fordi EU-subsidier gjorde det billigere å bruke pulver av skummetmelk som tilskudd i dyrefôr, så da gjorde man heller det.
\end{itemize}
Oljekrisen bidro også til at begge prosjektene endte uten suksess.

\paragraph{Mykoprotein}\index{mykoprotein} \ix{Quorn} er et vellykket matprodukt basert på det smak- og luktløse proteinet fra fungusen \idx{Fusarium} \emph{venenatum}. Men det kan transformeres til imitasjoner av fisk samt hvitt og rødt kjøtt fordi det består av omtrent det samme som kjøtt (50\% protein, 13\% fett, 25\% fiber), og man kan kontrollere lengden og finheten til fibrene ved å kontrollere hvor lenge man lar cellene vokse. En fordel med fungus i forhold til bakterieceller er at cellene gjerne er mye større og enklere kan separeres fra vekstmediet. \idx{Fusarium} gror i en blanding av glukosesyrup og ammoniakk.

\cstitle{Planter}\index{planter}

% \paragraph{Økning i produktivitet} På 60- og 70-tallet skjedde det en revolusjon i planteavl fordi nye og mer lønnsomme varianter av viktige planter som hvete og ris kom på banen. Samtidig økte bruken av gjødsel og pesticider.

\paragraph{Totipotente celler}\index{totipotens} Planteceller er totipotente, som betyr at man kan gro en helt ny plante fra en enkelt celle, så lenge man har de riktige planteveksthormonene. Disse hormonene inkluderer \idx{auxin}er, som regulerer vekst og differensiering, og \idx{cytokin}er, som fremmer vekst av skudd og forhindrer vekst av røtter. Forholdet mellom auxiner og cytokiner blir dermed en viktig størrelse.

\paragraph{Meristem}\index{meristem} Den delen av en plante som aktivt deler seg (altså der planten begynner på et utskudd). Det er her cellene er minst differensiert (analogt med stamceller i mennesker) Meristem kultur er bruk av celler fra meristem til å gro nye planter. Da skjærer man vekk en meristem og planter den på nytt i et vekstmedium.

Siden planteceller er totipotente, særlig ved meristemene, er det på denne måten mulig å lage opp til 500000 kloner av den samme planten i løpet av ett år. Ved å gro dem \idx{in vitro} (i testtuber med cellekultur) kan man raskt produsere nye varianter av den samme planten og avle frem egenskaper som resistans mot sykdommer og plantegift.

\paragraph{Haploide kulturer}\index{haploid} Pollenknappene (anthers) til planter er plantenes mannlige kjønnsceller, og inneholder kun ett sett kromosomer. Fruktknutene (ovaries) er de hunnlige kjønnscellene, og inneholder det andre settet med kromosomer. Slike celler som kun  inneholder ett av de to settene med kromosomer, kalles haploide (i motsetning til vanlige, diploide celler). Ved å gro celler fra pollenknapper i vekstmedium med hormoner kan man lage komplette, sterile planter der alle cellene er haploide. Disse er nyttige til planteavl fordi recessive mutasjoner blir synlige.

\paragraph{Callus}\index{callus} En ``celleklump'' med plantemateriale; den uorganiserte massen av celler som vokser der man har laget et kutt i en plante. Slik callus kan tas ut og gros videre i vekstmedium. 

\paragraph{Protoplast}\index{protoplast} Plantecelle der man har fjernet celleveggen (ved hjelp av cellenedbrytende enzymer som pektinaser og cellulaser). Når protoplaster vokser i vekstmedium vil de gjendanne celleveggene sine og danne calluser. Når plantehormoner dannes vil disse callusene få utskudd, og med andre hormoner vil disse utskuddene så slå røtter. Dermed kan man lage hele nye planter fra enkeltceller.

\paragraph{Somatisk hybridisering}\index{somatisk hybridisering} Siden protoplastene mangler cellevegg kan de relativt enkelt fusjoneres, enten med kjemisk cellefusjon eller med elektriske felt (elektrofusjon). Slike teknikker, som kalles somatisk hybridisering, ble brukt for å lage pomatplanter - hybdrider av potet og tomat. Pomatene var uspiselige og ingen stor suksess, fordi de forskjellige karakteristiske komponentene i de to plantene ikke fungerer godt sammen. Slike problemer vil svært sannsynligvis dukke opp når man prøver seg på en somatisk hybridisering.

\paragraph{Planteceller i bioreaktorer}\index{bioreaktor} Man trenger ikke differensierte plantedeler for å produsere plantemateriale i en reaktor - callus er nok, hvis man leverer det som trengs av inorganiske salter, vitaminer, sukker og hormoner, samt rister på løsningen for å få overført nok \ce{CO2} til plantene. Plantene vil da fortsette å gro, reprodusere og produsere planteprodukter i reaktoren. Planteceller i bioreaktorer brukes for å produsere mange ekstremt kompliserte forbindelser, særlig medisiner (steroider, kodein, atropin, reserpin, digoksin, digitoksin, quinin og Hyperzin A).

\paragraph{Suspensjonskultur}\index{suspensjonskultur} Det står ingenting om ``suspensjonskultur'' er, verken i Renneberg eller på slides, men det nevnes i forbindelse med protoplastkultur, så jeg antar at det er synonymt med det.

\paragraph{\idx{Agrobacterium}}\index{Ti-plasmid} En ``bakteriell genteknolog'' som brukes for å genmodifisere planter. Bakterien tiltrekkes av molekyler som planten slipper ut når den er skadet. Denne bakterien har et såkalt Ti-\ix{plasmid} (``Tumor-inducing plasmid'' fordi plasmidet forårsaker tumorvekst i planter). Ti-plasmidet har gener som koder for opin-detekterende proteiner. Opiner er spesielle aminosyrer som kun bakterien kan bruke, så bakterien ``reprogrammerer'' dermed plantecellen til sitt eget formål. I tillegg har Ti-plasmidet gener for proteiner som overfører DNA til skadede celler. Dette er et skjeldent eksempel på at gener overføres fra en prokaryot til en eukaryot. 

Ti-plasmider kan ``temmes'' ved å fjerne genet for opindetekterende proteiner og plantehormoner (det er de plantehormondannende genene som forårsaker tumorene). Samtidig setter man inn de genene man måtte ønske i Ti-plasmidet. Det er vanlig å inkludere et antibiotikaresistensgen, slik at rekombinante celler kan velges ut enkelt (ved å bruke vekstmedium med antibiotika som dreper de ikke-rekombinante bakteriene). De rekombinante plasmidene settes enten direkte inn i protoplaster, eller de settes inn i \emph{Agrobacterium} igjen (det sistnevnte er standardmetoden).

\paragraph{DNA-pistol}\index{DNA-pistol} En metode for å genmodifisere planter. Gull- og wolframpartikler dekkes med plasmid DNA, og partiklene skytes inn i vev på en måte som ikke skader cellene. Av og til integreres plasmid-DNAet i plante-DNAet selv om partiklene selv passerer gjennom cellene. Denne metoden kan også brukes for å sette inn DNA i kloroplaster, eller i andre organismer enn planter.

\paragraph{Eksempler på transgene planter}\index{transgen plante}
\begin{itemize}[nolistsep,noitemsep]
	\item \index{Round-Up}Planter kan gjøres resistante til glyfosfat ved å øke produksjonen av visse enzymer. Glyfosfat er en del av urgressmiddelet ``Round-Up'' som skader planter ved å hemme EPSP-syntase, et viktig enzym for aminosyremetabolisme. Genet for EPSP-syntase kan inkorporeres i rekombinant \idx{E. coli} og overføres til planter via \idx{Agrobacterium}. Dette gir planter med mye høyere konsentrasjon av EPSP, som gjør at de tolererer konsentrasjoner av ``Round-Up'' som dreper andre planter i området.
	\item Planter kan gjøres resistante til \ix{fosfinotricin} (PPT) med PPT acetyl transferase. Fosfinotricin dreper planter ved å hemme syntesen av glutamin, med den konsekvens at ammoniakk akkumulerer og dreper cellen. PPT acetyl transferase, et enzym i blant annet tobakk, poteter og raps, nøytraliserer denne hemmelsen.
	\item \emph{Luciferase}\index{luciferase}, enzymet som gjør ildfluer selvlysende, har blitt introdusert i tobakkplanter. Hvis det er luciferin og mye næringsstoffer i vannet til plantene (for eksempel hvis de vannes med avløpsvann), lyser de genmodifiserte plantene med en gulgrønn glød.
	\item \index{blå nellik}Man har klart å lage blå nellik ved å introdusere genene for det blå pigmentet i petunia. Blå roser har man dessverre ikke kommet helt i mål med.
	\item Tomater kan få tykkere skinn med mer stivelse (fint for å lage ketchup) ved å slå av polygalaktuonidase (PG) med \ix{antisense-RNA}.
	\item \index{raps}\index{soya}Transgen raps og soya har høyere innhold av lysin (som beskrevet tidligere er lysin en essensiell aminosyre, så høyere innhold av lysin betyr høyere næringsverdi). Transgen raps inneholder også et annet spekter av fettsyrer enn vanlig raps, blant annet noen av fettsyrene i palmeolje. Det kan derfor hende at transgen raps erstatter palmeolje.
	% \item Virusresistent sukkerbete blir ``vaksinert'' med BNYVV-viruset, som angriper bladene og reduserer sukkerinnholdet. Dette skjer ved at genetisk informasjon for et viralt protein settes inn i planten, slik at planten simulerer en respons på en virusinfeksjon. [WHAT?]
	\item \index{tørråte}Det gjøres forsøk på å lage planter som er resistante mot tørråtesopp. Slike planter produserer cellulase som angriper og løser opp celleveggen til tørråtesoppen.
	\item \index{vin}Genmodifiserte vindruer er et forsøk på å unngå de sykdomsfremkallende soppene som er et stort problem for tradisjonelle vindruesorter som Riesling, Merlot og Chardonnay.
	\item \index{golden rice}``Golden rice'' er ris som har blitt modifisert til å inneholde en høy konsentrasjon av $\beta$-karoten (provitamin A). Det er kjekt å ha, fordi vitamin A-mangel er vanlig blant folk som hovedsakelig lever på ris: 800 millioner mennesker lider av akutt vitamin A-mangel som forårsaker at synet, immunsystemet, blodet og skjelettveksten svekkes, og er en stor grunn til blindhet og anemi. Med de nyeste variantene av ``golden rice'' kan man dekke opp til halvparten av det daglige vitamin A-behovet ved å spise ris.
	\item Man prøver å lage koffeinfrie kaffeplanter. Sikkert kjekt for de 20 prosentene av verdens befolkning som drikker decaf-kaffe.
	\item Noen driver og lager poppeltrær som samler opp tungmetaller.
	\item Det gjøres forsøk på å lage planter som tåler tørke og høye konsentrasjoner av salt.
\end{itemize}

\paragraph{Biologisk insektmiddel} Insektmidler har en del ulemper: de dreper andre insekter enn insektmiddelet er beregnet på og forstyrrer dermed balansen i økosystemer, de akkumulerer i større fugler og dyr, og insektene blir etter hvert resistente (som tradisjonelt ``løses'' ved å bruke større mengde insektmiddel, slik at insektene etter hvert blir enda mer resistente, og så videre). 

Det biologiske insektmiddelet \ix{Bt-toksin} er et mikrobeprodusert krystallinsk protein som blir til gift i fordøyelsessystemet til larver og dermed perforerer dem fra innsiden. Bt-toksin er harmløst mot mennesker, fisk og varmblodige dyr. Ved å bruke de ovennevnte teknikkene for å inkorporere Bt-genet i planter får man et insektmiddel som kun påvirker insekter som er parasitter på planten.

\paragraph{``Refuge crops''}\index{refuge crops} Man ønsker å unngå spredningen av Bt-gener samt utviklingen og spredningen av Bt-resistente insekter. Dette gjøres ved å gro de transgene plantene ved siden av ikke-transgene planter. Dermed sørger man for at ikke-resistente insekter overlever og parrer seg med eventuelle resistente insekter; recessive resistensgener vil ikke komme til syne i avkommet til disse.

\paragraph{Trusler med transgene planter} Kan genmodifiserte planter spre seg til andre områder og bli til ugress der, på samme måte som kaniner ble til skadedyr da de ble introdusert til Australia? Trolig ikke: det ser ikke ut til at genmodifiserte planter kan overleve i naturen - det har i hvert fall aldri blitt vist at de kan det\footnote{Renneberg sier at det er slik ``as a general rule'', men forklarer ikke hvorfor. Nå er det jo slik at genene vi avler frem i transgene planter er til nytte for \emph{oss}, ikke for plantene selv. Det er derfor ikke noen grunn til å bekymre seg for at genmodifiserte planter får egenskaper som gjør dem mer levedyktige i naturen enn sine ikke-genmodifiserte søsterplanter. Se også \emph{Genetically Modified Foods Are Nothing New}, lenket til i Tonsill B.}.

Eller kan transgene planter overføre sine resistensgener til ugress gjennom krysspollinering, og dermed forverre ugressproblemet? Spørsmålet undersøkes for tiden.

En tredje bekymring er pollenspredning av genmodifiserte planter til ikke-genmodifiserte avlinger. Når det gjelder Bt-korn viser det seg at andelen genmodifiserte planter faller under 0.9\% (som er terskelen for når genmodifisert mat skal markeres i en del land, for eksempel Norge) når man er 10 meter unna et felt med Bt-planter. Derfor skal bønder som gror Bt-korn ha en tjue meter tykk separerende stripe rundt avlingen, for å utelukke økonomisk skade for nabobøndene.

\paragraph{Merking av genmodifisert mat} Skal \ix{genmodifisert mat} merkes? DNAet i seg selv har ikke noen metabolsk effekt, da det blir brutt ned raskt i fordøyelsessystemet. Hvis genmodifisert mat har noen effekt som gjør at den bør markeres, vil det være på grunn av proteinene som dannes av de innsatte genene, og det varierer naturligvis fra tilfelle til tilfelle. Dermed er ikke ``maten er genmodifisert'' i seg selv en god nok helsemessig grunn til å markere genmodifisert mat. Derfor er genmodifisert mat FDA-godkjent i USA.

Man kan riktignok argumentere for at den skal markeres av andre grunner enn helse, for eksempel for å hensynta enkeltpersoners religiøse innvendinger mot genmodifisering.

\paragraph{Regulering i Norge}\index{regulering i Norge} Med enkelte unntak gjelder de samme reglene i Norge som i andre europeiske land. Vi har matloven, som regulerer bearbeidede mat- og fôrprodukter som ikke inneholder levende, genmodifisert materiale, samt genteknologiloven om levende genmodifisert materiale (inkludert spiredyktige frø). Hittil har ingen genmodifiserte produkter blitt godkjent som mat eller fôr i Norge; før en eventuell godkjenning skal norske myndigheter ha gjennomført en risikovurdering. Dersom fôr- eller matvarer med mer enn 0.9\% genmodifisert materiale tilbys på markedet i Norge, skal den være merket.

\paragraph{Genopdrett}\index{genoppdrett} Planter er 10-50 ganger billigere å gro enn \idx{E. coli}, og 100 ganger billigere enn dyreceller. Det er også færre etiske kvalmer med å genmodifisere planter enn det er med å genmodifisere dyr, spesielt til medisinske formål. En annen viktig fordel med å bruke planter (og andre høyere organismer) til å gro gener, er at de kompliserte proteinene i høyere organismer modifiseres i cellen etter at de produseres (som vi så med insulin i kapittel 4).

Det har allerede blitt laget tobakk med antistoffer mot karies, og poteter og bananer med ``innebygd vaksine''. Slik mat, om den blir vellykket, kan ikke spises i for store doser og må derfor merkes tydelig, for eksempel ved å farge maten med en egen farge. Renneberg foreslår blå. Undertegnede er helt enig, og har alltid syntes at det er for lite blå mat på markedet.

\paragraph{Antifrost-bakterier}\index{antifrost-bakterier}\index{kunstig snø} Frost skader planter ved at iskrystaller dannes på bladene og bryter opp det levende vevet. Det viser seg at \idx{Pseudomonas} \emph{syringae}-bakterier spiller en viktig rolle i denne prosessen: de har ``frost-proteiner'' som stimulerer krystallvekst. Ved å populere planter med \emph{P. syringae}-bakterier der man har slettet DNAet som koder for frost-proteinene, beskyttes plantene fra frostskader fordi de skadelige naturlige bakteriene utkonkurreres av de harmløse genmodifiserte bakteriene.

% \footnote{Ifølge Renneberg kuttet man rett og slett vekk DNAet, men det bør vel også gå fint å bruke antisense-RNA.}

Samtidig brukes de ikke-genmanipulerte frostbakteriene til å produsere kunstig snø. De gros i \ix{bioreaktor}er og ødelegges (antagelig for å gjøre frost-proteinene tilgjengelige i løsning). Dette øker snøproduksjonen med 45\% og sparer energi som ellers ville blitt brukt på nedkjøling av naturlig snø.

\paragraph{Andre etiske problemer med genmodifisert mat} Potensielle økologiske trusler er nevnt i avsnittet om ``Trusler med transgene planter''. Samtidig kan store felt med \ix{genmodifisert mat} undergrave fornybart og alternativt jordbruk. Et problem som ikke er nevnt i Renneberg, men som er hintet til på slides, er at bruk av genmodifisert mat kan være mer lønnsomt for store bedrifter enn det er for små produsenter. Interesserte kan for eksempel lese seg opp på den kontroversielle businessmodellen til Monsanto, som er en viktig grunn til at bioteknologi og genteknologi har fått et dårligere omdømme enn teknologien kanskje fortjener.

% \footnote{ Interesserte kan for eksempel lese seg opp på den kontroversielle forretningspraksisen til gigantiske Monsanto.}
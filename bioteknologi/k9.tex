\bigskip\bigskip\bigskip % HACK
\ctitle{Medisinsk bioteknologi II}

\paragraph{Dette kapittelet} går litt i dybden på to forskjellige ting: stamceller, som er lovende verktøy for reparering av vev og behandling av sykdommer, og genterapi, som er modifisering av pasientens DNA for å behandle eller forebygge sykdommer.

\cstitle{Stamceller}

\paragraph{Stamcelle}\index{stamcelle}\index{pluripotens} Pluripotente celler, som kan spesialisere seg (gjennom \idx{differensiering}) til å bli forskjellige typer celler ved behov. Dette gjør dem til lovende verktøy for reparering av vev og behandling av sykdommer. Hvis stamceller transplanteres til et sykt organ, kan de lage det riktige erstattende vevet på stedet.

\paragraph{Typer stamceller} De tre viktigste typene stamceller er 
\begin{itemize}[nolistsep,noitemsep]
	\item Embryoniske stamceller (ESC): stamcellene i en blastocyst (et embryo på stadiet der det består av ca. 100 celler). Disse hentes fra ``overskudds-embryoer'' i forbindelse med kunstig befruktning. Fordeler med ESC er at de vokser lett i cellekultur, og at de kan bli til absolutt alle typer celler i kroppen. En annen stor fordel er at de ikke aldrer, men forblir friske og pluripotente gjennom mange runder med celledivisjon (fordi de har høye nivåer av telomerase, som bevarer lengden på telomerene). Ulemper med embryoniske stamceller er at man risikerer dannelse av teratoma (se under), etisk kontrovers rundt det å hente celler fra embryoer, og risiko for avvisning fra immunsystemet.
	\item Voksne stamceller: finnes i alt vev, men har begrenset levetid og vokseevne. Siden de ikke kan bli til alle typer celler i kroppen, men bare celler som finnes i vevet stamcellen kom fra, kalles de multipotente i stedet for pluripotente. Fordeler med bruk av voksne stammceller er at de er pasientens egne celler og dermed har mindre sannsynlighet for å bli avvist av kroppen, og man unngår det kontroversielle aspektet ved ESC. Problemer med bruk av voksne stamceller er at det finnes veldig få av dem, de har begrenset evne til å vokse \emph{in vitro}, og de er begrenset i hvilke celler de kan bli til.
	\item Induserte pluripotente stamceller (IPSC), som dannes fra spesialiserte voksne celler ved genetisk teknologi. Fordeler med slike celler er de samme som med voksne stamceller, men man er også mindre begrenset i hvilke celler man kan bruke, siden de pluripotente cellene kan blir til hva som helst. Utfordringer med IPSC er at de er vanskelige å produsere, resultatene man har oppnådd så langt er variable, man har ikke funnet ut hvilket potensial de har i forhold til ESC, og man risikerer i likhet med ESC dannelse av teratoma.
\end{itemize}

\paragraph{Teratom}\index{teratom} Tumor med vev eller organkomponenter som ligner på vanlig vev, altså ikke en krefttumor. Ukontrollert differensiering av stamceller kan føre til dannelse av teratoma. Derfor er det nødvendig med streng kontroll på selvfornyelsen og differensieringen til stamceller når man skal bruke dem terapeutisk.

\paragraph{Differensiering av stamceller} Differensiering av stamceller kontrolleres i kroppen av gener og epigenetiske forandringer, samt til tider av eksterne signaler som cytokiner og vekstfaktorer som slipper ut av celler i nærområdet, og fysisk kontakt med nabooceller og den ekstracellulære matriksen.

\emph{In vitro}\index{in vitro} kan differensiering av stamceller kontrolleres med faktorer som 
\begin{itemize}[nolistsep,noitemsep]
	\item komposisjonen til cellekulturen som stamcellene gror i
	\item overflatematerialet og -formen til beholderen man gror stamcellene i
	\item genmodifisering
\end{itemize}

\cstitle{Genterapi}

\paragraph{Genterapi}\index{genterapi} Modifisering av en organisme sitt DNA for å behandle eller forebygge sykdommer. Fordelen genterapi har over tradisjonell medisinbasert terapi, er at man løser de underliggende genetiske problemene som forårsaker sykdommen i stedet for å kun behandle symptomene. Innen genterapi finnes det 3 vanlige metoder:
\begin{itemize}[nolistsep,noitemsep]
	\item Erstatte et sykdomsfremkallende mutert gen med en frisk versjon av genet. Dette er den vanligste formen for genterapi.
	\item Inaktivere et mutert gen som fungerer feil.
	\item Introdusere et nytt gen i kroppen for å hjelpe til med å nedkjempe en sykdom.
\end{itemize}
De to første skal vi se nærmere på senere.

\paragraph{Eksempel: ADA-syndrom}\index{ADA-syndrom} Mangel på adenosin deaminase (ADA), som hvite blodceller trenger for å vokse og dele seg, fører til immunsvikt. Proteinet kan administreres til pasienten, men dette er ekstremt dyrt fordi det er en sjelden sykdom. Mangelen er forårsaket av en defekt i ett enkelt gen, og derfor klarte man i 1990 å behandle sykdommen med genterapi.

Prosedyren bestod i å injisere genet som koder for adenosin deaminase i et bakterielt plasmid (som beskrevet i kapittel 3), og så overføre plasmidet til et genetisk deaktivert retrovirus. Så lot man viruset angripe en kultur av T-celler fra pasienten, slik at retroviruset satt inn det fungerende ADA-genet i T-cellene. De virusinfiserte T-cellene, som nå kunne produsere adenosin deaminase, ble så sprøytet inn i pasienten igjen.

\paragraph{Genvektor}\index{genvektor} Mekanismen man bruker for å smugle et gen inn i en celle. En god vektor
\begin{itemize}[nolistsep,noitemsep]
	\item går inn i de riktige cellene
	\item integrerer genet i cellen
	\item aktiverer genet
	\item har ikke skadelige bivirkninger
\end{itemize}

\paragraph{Virale vektorer}\index{viral vektor} Virus er fra naturens side allerede maskiner som er laget for å sette inn gener i celler. Fordeler med å bruke virus som vektorer er at de er gode på å finne og infiltrere celler, noen av dem går målrettet på bestemte typer celler, og de kan modifiseres slik at de ikke repliserer og destruerer celler. Problemer med å bruke virus som vektorer er at de er begrenset i hvor mye genetisk materiale de kan ta med seg (fordi de er små), og at de kan forårsake immunrespons i pasienter slik at pasienten enten blir syk eller forhindrer viruset fra å levere genet til cellene.

\paragraph{Ikke-virale vektorer}\index{ikke-viral vektor} Andre måter å levere gener på, er
\begin{itemize}[noitemsep,nolistsep]
	\item Plasmider.\index{plasmid} Disse kan være pakket inn i liposomer (små membranbundne pakker som leverer innholdet sitt ved å smelte sammen med cellemembranen). En fordel er at de kan ta med seg store gener, og de fleste forårsaker ikke en immunrespons. En ulempe er at de er mye mindre effektive enn virus for å overføre gener til celler. 
	\item \emph{Virosomer}:\index{virosom} \ix{liposom}er dekket med virale overflateproteiner (disse proteinene hjelper virosomet med å trenge inn i cellen, og kan også bidra til spesifisitet). Disse kombinerer plasmiders evne til å ta med seg store gener og unngå immunsystemet med virusenes spesifisitet.
	\item Polykationer: positivt ladde polymere som danner komplekser med det negativt ladde DNAet for å danne en partikkel som kan spises av cellen (fagocytose).\index{polykation}\index{fagocytose}
\end{itemize}

\paragraph{\emph{Ex vivo} og \idx{in vivo} genterapi} Gener kan enten leveres til pasienten \emph{in vivo}, ved at pasienten injiseres direkte med vektoren, eller \idx{ex vivo}, ved at genet leveres til celler som har blitt tatt ut av kroppen og gror i en cellekultur, og disse cellene settes så inn i pasienten igjen. \emph{Ex vivo} genterapi har mindre sjanse for å trigge en immunrespons fordi man ikke setter inn virus i pasienten, samtidig som man kan følge med på cellene og se at de fungerer som de skal før de settes inn i pasienten igjen.

\paragraph{Reparasjon av mutasjoner}\index{mutasjon}\index{SMaRT} Det har blitt utviklet forskjellige virale vektorer som reparerer mutasjoner direkte i DNAet. Disse bruker enzymer som målrettet går inn på spesifikke DNA-sekvenser, kutter ut den defekte sekvensen og setter inn en fungerende kopi. En annen teknologi, SMaRT (``Spliceosome-Mediated RNA Trans-splicing'') retter seg mot mRNA-et som har blitt transkribert fra det muterte genet, og reparerer den delen av mRNAet som inneholder mutasjonen.

\paragraph{Gene silencing}\index{gene silencing} Hvis det er like greit at man ikke produserer det feilaktige genet i det hele tatt, finnes det forskjellige teknikker for å ``slå av'' genet slik at det ikke produseres noe protein fra det. Dette kan gjøres med oligonukleotider som binder seg til åpningen mellom de to DNA-strengene og blokkerer DNAet fra å transkriberes til mRNA. 

Det går også an å påvirke mRNAet etter at det har blitt transkribert. En måte å gjøre dette på, er å introdusere en liten bit RNA med en nukleotidsekvens som er komplementær til mRNAet som ble transkribert fra genet man ønsker å slå av. Det to bitene RNA vil binde seg til hverandre og lage en ``ubrukelig'' RNA/RNA-hybrid som cellen destruerer (dette kalles \idx{antisense-RNA}, og ble brukt for å motvirke virus i kapittel 5). En annen metode kalles ribozym-genterapi, der man designer RNA-enzymer (ribozymer) som finner og destruerer mRNA som er transkribert fra det muterte genet.\index{ribozym}

Denne typen RNA-interferens kan også brukes analytisk: ved å slå av gener på en systematisk måte kan man finne ut hvilke gener som fremkaller hvilken effekt når de uttrykkes.

\paragraph{SiRNA}\index{SiRNA} ``Small interfering RNA'', små RNA-tråder på 21-23 nukleotider, som binder seg til mRNA og lager dobbeltstreng-RNA som destrueres av cellen. Så lenge RNA-trådene er korte nok, slipper de unna immunsystemet. Dette er den mest effektive måten man har for å slå av individuelle gener uten immunrespons.

SiRNA er det samme som \ix{antisense-RNA}. Ikke la noen fortelle deg noe annet. Eller, SiRNA er et subsett av antisense-RNA, det subsettet av antisense-RNA som faktisk fungerer.

\paragraph{Genterapi i fremtiden}\index{genterapi} Det er omtrent 5000 gener som er av interesse i genterapi-sammenheng. Genterapi virker lovende for en del genetiske sykdommer, noen typer kreft og enkelte virusinfeksjoner, men det er fortsatt uvisst om teknikken er trygg og effektiv nok til å kunne bli vanlig. Genterapi testes for tiden kun på sykdommer som ikke har noen andre kjente kurer.

I 2012 ble den første genterapibehandlingen akseptert for klinisk bruk i Europa og USA; \ix{Glybera} er en genterapi-behandling som kompenserer for mangel på lipoprotein-lipase, som er nødvendig for effektiv nedbryting av fettsyrer.

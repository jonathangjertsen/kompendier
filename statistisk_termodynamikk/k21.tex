\ctitle{Elektrostatisk potensial}
\paragraph{Dette kapittelet}
En mer ``standard'' behandling av elektrostatikk gir de samme resultatene. Vi går veldig fort gjennom dette, fordi det ikke er noe her som ikke er kjent fra et grunnkurs i elektromagnetisme.

\cstitle{Hva er elektrostatisk potensial?}
Arbeidet $\delta W$ som kreves for å bevege en ladning $q$ en liten distanse $d\vec{l}$ i et konstant elektrisk felt $\vec{E}$ er 
\begin{equation}
	\delta w = -\vec{f}\cdot d\vec{l}=-q\vec{E}\cdot d\vec{l}
\end{equation}
der minustegnet indikerer at arbeidet gjøres \emph{mot} feltet, og ikke av feltet.
For å bevege ladningen fra A til B kreves derfor
\begin{equation}
	w_{AB}=-q\int_{A}^B \vec{E}\cdot d\vec{l}
\end{equation}
Vi \emph{definerer} nå potensialforskjellen $\Delta \psi = \psi_{AB}$ som arbeidet som kreves for å bevege en testladning med størrelse 1 fra A til B:
\begin{equation}
	\Delta \psi = \psi_{AB}=\psi_B-\psi_A=\frac{W_{AB}}{q_{test}}=-\int_A^B\vec{E}\cdot d\vec{l}
\end{equation}
så det elektrostatiske potensialet ganger en ladning blir en energi.
Integralligningen er ekvivalent med differensialligningen i \eqref{enabla}.

\cstitle{Elektrostatiske interaksjoner er konservative}
Elektrostatisk arbeid er reversibelt og derfor veiuavhengig. Arbeidet som utføres for å flytte en ladning fra et punkt $A$ til et punkt $B$ via en bane $\mathscr{C}$. Videre summerer det elektrostatiske arbeidet opp til 0 dersom banen går tilbake til utgangspunktet.

\cstitle{Poissons ligning}
Er:
\begin{equation}
	\nabla^2\psi=-\frac{\rho}{\epsilon_0D}
\end{equation}
Merk at
\begin{equation}
	\nabla\cdot\vec{E}=-\nabla^2\psi.
\end{equation}
Det setter vi pris på.

\ctitle{Solvatering og transport av molekyler mellom faser}
\paragraph{Dette kapittelet} Det kjemiske potensialet beskriver molekylers tendens til å dele seg opp i faser, og til overføring av molekyler mellom faser. Mange egenskaper er \i{kolligative egenskaper}: de er avhengige av antall molekyler løst stoff i løsemiddelet. Noen slike egenskaper er hvordan damptrykket senkes av oppløste stoffer, hvorfor salt smelter is og kondenserer kokende vann, osmotisk trykk, partisjonskoeffisienter og dimerisering av et løsemiddel.

Vi ser på et system av to komponenter der komponenten med høy konsentrasjon kalles løsemiddelet og det med lav konsentrasjon er det oppløste stoffet.

Vi har to konkurrerende drivkrefter:
\begin{itemize}
	\item Molekyler vil bevege seg fra områder med høy konsentrasjon til områder med lav konsentrasjon for å \emph{øke entropien}.
	\item Molekyler vil bevege seg til områder der de har høy kjemisk affinitet for å \emph{senke den frie energien}.
\end{itemize}

\cstitle{Solvatering}
Solvatering er overføringen av molekyler mellom gass- og væskefaser. Modellsystemet vårt er en blanding av to komponenter $A$ og $B$, der $B$ er flyktig ($B$-molekyler kan bytte mellom gass- og væskefasen) og $A$ ikke er det ($A$-molekyler holder seg til væskefasen). Vi bruker det isobar-isoterme ensemblet (konstant $(T,p,N_A,N_B)$, Gibbs fri energi er minimert), samt at $N_B^{\text{gass}}+N_B^{\text{væske}}=N_B$. Ved konstant $p$ og $T$ er likevektsbetingelsen vår, som utledet tidligere, at
\begin{equation}
	\label{muquilibrium}
	\mu_B^{\text{gass}}=\mu_B^{\text{væske}}
\end{equation}

\paragraph{En gittermodell for solvatering} For en ideell gass er 
\begin{equation}
	\mu_B^{\text{gass}}=k_BT\ln\frac{p_B}{p_{B,\text{int}}^o}
\end{equation}
og for blandingen har vi funnet ut at
\begin{equation}
	\mu_B^{\text{væske}}=k_BT\ln x_B+\frac{zw_{BB}}{2}+k_BT\chi_{AB}(1-x_B)^2
\end{equation}
dermed får vi ved likevekt \eqref{muquilibrium} at
\begin{align}
	\ln\frac{p_B}{p_{B,\text{int}}^o}&=\ln x_B +\frac{zw_{BB}}{2k_BT}+\chi_{AB}(1-x_B)^2 \\
	\frac{p_B}{p_{B,\text{int}}^o}&=x_B \exp\left(\chi_{AB}(1-x_B)^2+\frac{zw_{BB}}{2k_BT}\right)
\end{align}
som vi kan skrive om til (legg merke til hvordan uttrykket nedenfor ligner på Bolzmannfaktorer når man husker at $\chi_{AB}$ inneholder et $1/k_BT$-ledd)
\begin{equation}
	\label{vaporpressurek16}
	p_B=p_B^ox_Be^{\chi_{AB}(1-x_B)^2},
\end{equation}
der
\begin{equation}
	p_B^o = p_{B,\text{int}}e^{\frac{zw_{BB}}{2k_BT}}
\end{equation}
er damptrykket til det rene stoffet, som vi så i \eqref{vaporpressure}. $p_B$ i \eqref{vaporpressurek16} er damptrykket til løsningen av $A$- og $B$-molekyler. Se figur 16.1 i boka for å se hvordan dette damptrykket avhenger av $\chi_{AB}$.

\cstitle{To ekstreme situasjoner}
To eksempler som vi skal se på er natriumkloridsalt ($A$) oppløst i vann ($B$) og karbondioksidgass ($B$) oppløst i vann ($A$).

\paragraph{Situasjon 1: salt i vann} Her er det løsemiddelet vann som er den flyktigste komponenten, så vann noteres som $B$. Da er $x_B\approx 1$ fordi det er mye mer vann enn salt, så \eqref{vaporpressurek16} blir til $p_B=p_B^ox_B$ - som er det samme uttrykket vi hadde fått hvis vi antok en ideell løsning. Uttrykket kalles \i{Raoults lov}. Ved å legge til salt synker $x_B$, så damptrykket til saltvann er lavere enn damptrykket til rent vann. Siden vi har samme situasjon som for en ideell løsning, tolker vi denne senkingen av damptrykket som en entropisk effekt; det finnes flere unike måter å organisere vannmolekyler i en saltvannsløsning enn i en løsning rent vann, så salt vil gå inn og trykke ut vannet fra løsningen.

\paragraph{Situasjon 2: karbondioksid i vann} Her er det karbondioksid som er den relativt flyktige komponenten og noteres $B$. Da er $x_B\approx0$ siden det er mye mer vann enn det er karbondioksid, så \eqref{vaporpressurek16} blir til $p_B=p_B^oe^{\chi_{AB}}x_B$, som er \i{Henrys lov}. Dette tolkes som en energetisk effekt, se neste delkapittel. Dette resultatet brukes i laboratorieeksperimenter for å måle $\chi_{AB}$ eksperimentelt (ved å variere $x_B$).

\cstitle{Henrys lov}
I literaturen kalles $k_H=p_{B,\text{int}}^oe^{\chi_{AB}}$ Henrys konstant. Hvis vi utvider $\chi_{AB}$ ser vi at
\begin{align}
\begin{split}
	k_H&=p_{B,\text{int}}^o\exp\left(\frac{1}{k_BT}\left(zw_{AB}-\frac{1}{2}zw_{AA}\right)\right) \\
	&=p_{B,\text{int}}^o\exp\left(\frac{\Delta h_{\text{løsning}}^o}{k_BT}\right)
\end{split}
\end{align}
Løsningsentalpien $\Delta h_{\text{løsning}}^o$ er den frie energien involvert i å fjerne et $A$-molekyl fra et gitter av $A$-molekyler (slik at man bryter $z/2$ $AA$-bindinger) og så sette inn et $B$-molekyl (slik at man danner $z$ $AB$-bindinger).

Dette viser at $B$ kommer til å løse seg i $A$ dersom enten $w_{AB}$ er sterk (i så fall vil det være energetisk fordelaktig å danne $AB$-bindinger) eller $w_{AA}$ er svak (da koster det så lite å bryte $AA$-bindinger at entropiske effekter lett kan overkomme energibarrieren assosiert med dette).

\cstitle{Oppløste stoffer øker kokepunktet til løsemiddelet}
Kokepunktet $T_b$ er temperaturen der damptrykket er lik det atmosfæriske trykket. La oss føst anta at vi har en ren væske med kokepunkt $T_{b0}$ ved et trykk $p_{\text{atm}}$, slik at \eqref{vaporpressurek16} reduseres til
\begin{equation}
	p_{\text{atm}}=p_{b,\text{int}}^o\exp\left(\frac{zw_{BB}}{2RT_{b0}}\right)
\end{equation}
La oss så anta at vi har en løsning med saltkonsentrasjon $x_A$ ved sitt eget kokepunkt $T_{b1}$, som også står ved et trykk $p_{\text{atm.}}$. Med en ideell løsning som følger Raoults lov får vi at \begin{equation}
	p_{\text{atm}}=p_{b,\text{int}}^ox_B\exp\left(\frac{zw_{BB}}{2RT_{b1}}\right)
\end{equation}
Settes disse like hverandre får vi at 
\begin{equation}
	\ln x_B=\frac{zw_{BB}}{2R}\left(1/T_{b0}-1/T_{b1}\right)
\end{equation}
En Taylorekspansjon rundt $T_{b0}$ gir et nyttigere uttrykk for $\left(1/T_{b0}-1/T_{b1}\right)$:
\begin{equation}
	\left(1/T_{b0}-1/T_{b1}\right)\approx1/T_{b0}+(T_{b1}-T_{b0})\frac{-1}{T_{b0}^2}=\frac{\Delta T}{T_{b0}^2}
\end{equation}
I tillegg har vi at 
\begin{equation}
	\ln x_B = \ln (1-x_A) \approx -x_A - \frac{x_A^2}{2} - \frac{x_A^3}{3} - ...
\end{equation}
som vi forkorter til det første leddet i rekka.

Til slutt introduserer vi
\begin{equation}
	\label{vapenthalpy}
	\Delta h_{\text{vap}}^o = \Delta h_{\text{gas}}^o - \Delta h_{\text{liq}}^o = \frac{-zw_{BB}}{2}
\end{equation}
for å få det vanlige uttrykket i literaturen:
\begin{equation}
	\Delta T = \frac{RT_{b0}^2x_A}{\Delta h_{\text{vap}}^o}
\end{equation}
Modellen vår hjelper oss med å forklare hvorfor $\Delta T>0$, dvs. hvorfor kokepunktet øker når vi legger til salt. $R$, $T_{b0}^2$ og $x_A$ er positive, men $\Delta h_{\text{vap}}^o$ er også positiv fordi den avhenger av de attraktive kreftene $w_{BB}$ som beskrevet i \eqref{vapenthalpy}. Dette gjelder for en ideell løsning og kan tolkes som en entropisk effekt - ved å legge til salt øker man den maksimale entropien til løsningen slik at endringen i entropi ved å gå over til gassfase blir mindr, så de attraktive interaksjonskreftene kan holde sammen løsningen ved en litt høyere temperatur.

%\cstitle{Oppløste stoffer senker frysepunktet til løsemiddelet}
%Hvis saltet bare er tilstede i vann i væskefase og ikke i is, får vi tilsvarende uttrykk: \footnote{11/21 - der har man lagt til correction terms fra taylorekspansjonen}. Dette kan igjen forklares som en entropisk effekt.

\cstitle{Konsentrasjon}
Noen ganger bruker vi andre enheter enn molfraksjoner $x_A=\frac{n_A}{n_A+n_B}\approx\frac{n_A}{n_B}$: \i{molaritet}en er $c_A=\frac{n_A}{V}$ og \i{molalitet}en er $m_A\approx\frac{x_A}{M_B}$ der $M_B$ er molar masse. Hvis man kun inkluderer det lineære leddet i endringen av kokepunktet får man at 
\begin{equation}
	\Delta T = K_bm_A=\frac{RT_b^2M_B}{\Delta h_{\text{vap}}^o}m_A,
\end{equation} der $K_b$ kun avhenger av egenskapene til det rene løsemiddelet.

\cstitle{Osmotisk trykk}
Modellsystemet vårt er en ren væske $B$ som er adskilt fra en blanding av $A$ og $B$ ved en halvgjennomtrengelig membran der $B$ kan slippe gjennom, men ikke $A$. $B$-molekyler vil trekkes fra den rene væsken til blandingen for å øke entropien (det kan hende entropien til blandingen synker, men totalt sett vil entropien øke for det samlede systemet).

Ved å akkumulere et ekstra volum på toppen av blandingen får vi et hydrostatisk trykk $\pi$ som motvirker effekten beskrevet i forrige avsnitt. $\pi$ kalles \i{osmotisk trykk}.

Likevektsbetingelsen vår er at 
\begin{equation}
	\label{yetanotherequilibrium}
	\mu_B^{\text{rent stoff}}(p)=\mu_B^{\text{blanding}}(p+\pi,x_B)
\end{equation}
Siden fri energi er en tilstandsfunksjon kan vi dele opp prosessen i to trinn: ett hvor vi kun øker trykket, og ett hvor vi kun legger til $A$-molekyler.

\paragraph{Økning av trykk ved konstant temperatur} I første trinn er 
\begin{equation}
	\mu_B^{\text{rent stoff}}(p+\pi) = \mu_B^{\text{rent stoff}}(p) + \int_{p}^{p+\pi}\spdif{\mu_B}{p}\dpp.
\end{equation}
Vi kan bruke Maxwell-relasjonen
\begin{equation}
	\spdif{\mu_B}{p}=\spdif{V}{N_B}=v_B
\end{equation}
for å få at
\begin{align}
\begin{split}
	\mu_B^{\text{rent stoff}}(p+\pi) &= \mu_B^{\text{rent stoff}}(p) + \int_{p}^{p+\pi}v_B\dpp \\
	&=\mu_B^{\text{rent stoff}}(p) + \pi v_B.
\end{split}
\end{align}

\paragraph{Tilsetning av oppløst stoff ved konstant trykk} I andre trinn gjelder \eqref{bullshit} (for en fortynnet løsning er $\gamma_B\approx 1$), så
\begin{equation}
	\mu_B^{\text{blanding}}(p+\pi,x_B) = \mu_B^{\text{rent stoff}}(p+\pi) + RT\ln(\gamma_B x_B)
\end{equation}

\paragraph{Ved likevekt} blir dermed \eqref{yetanotherequilibrium} til
\begin{equation}
	-\pi v_B = RT\ln x_B
\end{equation}
Hvis vi så bruker Taylorutvidelsen
\begin{equation}
	\ln x_B = \ln (1-x_A) \approx -x_A - \frac{x_A^2}{2} - \frac{x_A^3}{3} - ...
\end{equation}
og bare tar med det første leddet, får vi ``lærebok-resultatet''
\begin{equation}
	\pi = \frac{RT}{v_B}x_A\approx\frac{n_ART}{V}=c_ART,
\end{equation}
som vi kan utvide systematisk ved å ta med flere ledd i Taylorutvidelsen.

\cstitle{Partisjonskoeffisienter}
La oss så se på et system med to ublandbare løsemidler $A$ og $B$ (for eksempel olje og vann), der et løsemiddel $s$ kan bevege seg mellom de to fasene. Partisjonskoeffisienten $K_A^B$ er da definert som 
\begin{equation}
	\label{partitioncoefficient}
	K_A^B=x_s^B/x_S^A,
\end{equation}
der $x_s^A$ er molfraksjonen av $s$ i løsemiddelet $A$. 

Likevektsbetingelsen er stadig at kjemisk potensial er det samme overalt, her betyr det at det kjemiske potensialet er det samme i hver av de to fasene,
\begin{equation}
	\mu_s^A=\mu_s^B.
\end{equation}
Vi setter inn \eqref{anotherchemicalpotentialA} og \eqref{anotherchemicalpotentialB} for å få at
\begin{equation}
	\frac{zw_{ss}}{2k_BT}+\ln x_s^A+\chi_{sB}(1-x_s^A)^2 = \frac{zw_{ss}}{2k_BT}+\ln x_s^B+\chi_{sB}(1-x_s^B)^2, 
\end{equation}
og
\begin{equation}
	\ln K_A^B = \ln x_s^B/x_S^A = \chi_{sA}(1-x_s^A)^2-\chi_{sB}(1-x_s^B)^2
\end{equation}
så tilnærmer vi ved å bruke at $x_s^A \ll 1$ og $x_s^B \ll 1$ for å få det endelige resultatet
\begin{equation}
	\ln K_A^B \approx \chi_{sA}-\chi_{sB}.
\end{equation}

\cstitle{Dimerisering}
La oss til slutt se på likevekten mellom en solvatert dimer $AB$ og to solvaterte monomerer $A$ og $B$ i samme løsning av $s$-molekyler. Likevektskonstanten for reaksjonen \ce{A + B <-> AB} er 
\begin{equation}
	K_{\text{dimer}}=\frac{x_{AB}}{x_Ax_B}
\end{equation}
Vi gjør så en tilnærming ved å anta at $A$, $B$ og $AB$ er \emph{uendelig fortynnet} i løsemiddelet, slik at hvert molekyl av $A$, $B$ eller $AB$ kun interagerer med $s$-molekyler, og ikke med andre $A$-, $B$- eller $AB$-molekyler.
Ved konstant trykk og temperatur er
\begin{equation}
	\dG=\mu_A\dN_A+\mu_B\dN_B+\mu_{AB}\dN_{AB}=0
\end{equation}
Massebevaring gir at
\begin{equation}
	\dN_{AB}=-\dN_A=-\dN_B,
\end{equation}
slik at
\begin{equation}
	(\mu_A-\mu_B+\mu_{AB})\dN_{AB}=0,
\end{equation}
altså er 
\begin{equation}
	\mu_{AB}=\mu_A+\mu_B.
\end{equation}
som innsatt \eqref{bullshit} og \eqref{gammaprox} blir til
\begin{equation}
	\left(\frac{\mu_{AB}^o}{k_BT}+\ln x_{AB}\right)=\left(\frac{\mu_{A}^o}{k_BT}+\ln x_{A}\right)+\left(\frac{\mu_{B}^o}{k_BT}+\ln x_{B}\right),
\end{equation}
slik at
\begin{align}
\label{baaaaa}
\begin{split}
	\ln K_{\text{dimer}}&=\ln\frac{x_{AB}}{x_Ax_B} \\
	&=-\frac{1}{k_BT}(\mu_{AB}^o-\mu_A^o-\mu_B^o) \\
	&=-\frac{\Delta \mu^o}{k_BT}
\end{split}
\end{align}
La oss se litt nærmere på $\mu_A^o$ (og $\mu_B^o$). Vi kan utvide \eqref{anotherchemicalpotentialA} til
\begin{align}
\begin{split}
	\frac{\mu_A^o}{k_BT}&=\frac{zw_{AA}}{2}+\chi_{sA}(1-x_A)^2-\ln q_A \\
	&\approx \frac{zw_{AA}}{2}+\chi_{sA}-\ln q_A \\
	&=\frac{z}{k_BT}\left(w_{sA}-\frac{w_{ss}}{2}\right)-\ln q_A
\end{split}
\end{align}
der vi har tilnærmet med $\chi_{sA}(1-x_A)^2\approx\chi_{sA}$ i en uendelig fortynnet løsning, og vi har inkludert partisjonsfunksjonen. Dette har vi gjort fordi dimeriseringen forandrer frihetsgrader assosiert med rotasjon og vibrasjon - i kapittel 15 så vi bare på reaksjoner der disse frihetsgradene var bevart slik at vi kunne se bort i fra bidrag fra rotasjons- og vibrasjonsfrihetsgrader til partisjonsfunksjonen (siden disse bidragene ikke endret seg i løpet av reaksjonen), men nå kan vi altså ikke se bort i fra slikt.

Så må vi finne $\mu_{AB}^o$ for å få et uttrykk for $\Delta \mu^o$. $\mu_{AB}^o$ er den frie energien man får fra en prosess der man fjerner to $s$-molekyler og setter inn en $AB$-dimer i et gitter av $s$-molekyler (normalisert med $k_BT$). Denne energien finner vi med følgende ressonement: 
\begin{enumerate}[nolistsep,noitemsep]
	\item Du et hulrom ved å fjerne to tilstøtende $s$-molekyler. Disse to $s$-molekylene hadde til sammen $2(z-1)$ naboer (siden en av naboene var det andre $s$-molekylet), så ved å lage hulrommet har du brutt $\frac{2(z-1)}{2}=z-1$ bindinger. Derfor er kostnaden assosiert med en slik prosess lik $(-w_{ss})(z-1)$.
	\item Så setter du inn ett $A$-molekyl og ett $B$-molekyl i hulrommet. Da får du en energi $(z-1)w_{sA}$ fra å sette inn $A$-molekylet, en energi $(z-1)w_{sB}$ fra å sette inn $B$-molekylet, og en energi $w_{AB}$ fra bindingen mellom $A$ og $B$-molekylet.
	\item Samtidig må du slenge på et bidrag $k_BT\ln q_{AB}$ for å ta hensyn til endring av rotasjons- og vibrasjonsfrihetsgrader.
\end{enumerate}
Dermed blir
\begin{equation}
	\frac{\mu_{AB}^o}{k_BT}=\frac{z-1}{k_BT}\left(w_{sA}+w_{sB}-w_{ss}\right)+\frac{w_{AB}}{k_BT}-\ln q_{AB}.
\end{equation}
så setter vi inn i \eqref{baaaaa}, 
\begin{align}
\begin{split}
	\ln K_{\text{dimer}} &= -\frac{\Delta \mu^o}{k_BT} \\
	&=\frac{1}{k_BT}(w_{sA}+w_{sB}-w_{ss}-w_{AB})+\ln\frac{q_{AB}}{q_Aq_B}\\
	&=\frac{1}{z}\left(\chi_{sA}+\chi_{sB}-\chi{AB}\right)+\ln\frac{q_{AB}}{q_Aq_B}
\end{split}
\end{align}
Litt visdom som ligger i dette uttrykket, er:
\begin{enumerate}[nolistsep,noitemsep]
	\item Forskjellen i partisjonsfunksjoner kan utgjøre et substansielt bidrag til likevektskonstanten, siden $q_A$ og $q_B$ typisk bare inkluderer translasjon, mens $q_{AB}$ også inneholder bidrag fra vibrasjon og rotasjon.
	\item Sterke interaksjoner mellom $A$- og $B$-molekyler vil drive likevekten mot dimerisering.
	\item Sterke interaksjoner mellom $A$- og $s$-molekyler, eller mellom $B$- og $s$-molekyler, vil drive likevekten vekk fra dimerisering.
	\item Sterke interaksjoner mellom $s$- og $s$-molekyler vil også drive likevekten mot dimerisering. Et eksempel på dette er den hydrofobiske effekten - dimerisering av to upolare molekyler i vann - som forårsakes av en stor $w_{ss}$.
\end{enumerate}

Med dette har vi nådd høyden av kompleksitet i TKJ4215. Vi har bygget opp et ``termodynamisk maskineri'' som kan brukes på mange forskjellige fysiske fenomener, som er det vi kommer til å gjøre i de resterende kapitlene. Kapitlene i del 2 er derfor mindre sammenhengende enn kapitlene i del 1, og det er ikke nødvendig å gå gjennom dem i noen bestemt rekkefølge.
%\pagebreak
